\documentclass[a4paper]{ltjsarticle}
%\usepackage[utf8x]{inputenc}
%\usepackage[T1]{fontenc}
\usepackage{tikz}
\usepackage{luacode}
\usepackage[box,completemulti,lang=JA]{automultiplechoice}
\newcommand*{\var}[1]{\luadirect{tex.print(#1)}}
\begin{document}
\AMCboxDimensions{shape=oval,width=1.8ex,height=2.5ex}
\AMCcodeVspace=0em
\directlua{math.randomseed(220511)} % \onecopy{} の外に配置

\begin{luacode*}
function execSage(command)
  local sagecommand="sage -c 'print(latex("..command.."))'"
  local handle = io.popen(sagecommand, "r")
  local content=string.gsub(handle:read("*all"), "\n", "")
  handle:close()
  return content
end
function execMaxima(command)
  local maximacommand="echo 'tex1("..command..");'| maxima --very-quiet"
  local handle = io.popen(maximacommand, "r")
  local content=string.gsub(handle:read("*all"), "\n", "")
  handle:close()
  return content
end
\end{luacode*}

\onecopy{2}{
%%% 試験問題用紙ヘッダー開始:    

\noindent{\bf 応用数学 演習04 \hfill 2022年5月11日}

\vspace{3ex}

{\small
  \setlength{\parindent}{0pt}\hspace*{\fill}\AMCcodeGridInt{学生番号}{8}\hspace*{\fill}
\begin{minipage}[b]{6.5cm}$\longleftarrow{}$\hspace{0ptplus1cm}
学生番号を左にマークし、下に氏名を記入してください。

\vspace{3ex}

\hfill\namefield{
  \fbox{
    \begin{minipage}{.9\linewidth}
      氏名:\vspace*{.5cm}\dotfill\vspace*{.5cm}\dotfill
      \vspace*{1mm}
    \end{minipage}
  }
}
\hfill\vspace{5ex}
\end{minipage}
\hspace*{\fill}
}

\vspace{1ex}
\hrulefill
\vspace{1ex}

%%% ヘッダー終了

%%%%%%%%%%%%%%%%%%%%%%%%%%%%%%%%%%%%%%%%%%%%%%%%%%%%%%%%%%%%%
\begin{question}{diff09}
\luadirect{
  a=math.random(2, 4);
  b=math.random(5, 9);
  c=math.random(2, 9);
  s=-1
  t=math.tointeger((-1)^(math.random(0,1)));
  g='('..a..'*'..s..'*x^2+'..t..'*'..b..')';
  f=g..'^'..c;
  h='diff(g, x)^'..c
  formula=execMaxima(f);
  correct1=execMaxima("diff("..f..", x)");
  wrong1=execMaxima("diff((1/2)*"..f..", x)");
  wrong2=execMaxima("diff((-1/2)*"..f..", x)");
  wrong3=execMaxima("diff((-1)*"..f..", x)");
  wrong4=execMaxima(c.."*("..g.."^("..c.."-1))");
}
函数 \(f(x)=\var{formula}\)の導函数\(f'(x)\)を求めなさい.
  \begin{choiceshoriz}
    \correctchoice{\(\var{correct1}\)}
    \wrongchoice{\(\var{wrong1}\)}
    \wrongchoice{\(\var{wrong2}\)}
    \wrongchoice{\(\var{wrong3}\)}
    \wrongchoice{\(\var{wrong4}\)}
  \end{choiceshoriz}
\end{question}

%%%%%%%%%%%%%%%%%%%%%%%%%%%%%%%%%%%%%%%%%%%%%%%%%%%%%%%%%%%%%
\begin{question}{diff10}
\luadirect{
  s=math.tointeger((-1)^(math.random(0,1)));
  f="sqrt(x^2+"..s..")";
  formula=execMaxima(f);
  correct1=execMaxima("diff("..f..", x)");
  wrong1=execMaxima("diff(2*"..f..", x)");
  wrong2=execMaxima("diff((-2)*"..f..", x)");
}
函数 \(f(x)=\var{formula}\)の導函数\(f'(x)\)を求めなさい.
  \begin{choiceshoriz}
    \correctchoice{\(\var{correct1}\)}
    \wrongchoice{\(\var{wrong1}\)}
    \wrongchoice{\(\var{wrong2}\)}
    \wrongchoice{\(\sqrt{2x}\)}
    \wrongchoice{\(1\)}
  \end{choiceshoriz}
\end{question}

%%%%%%%%%%%%%%%%%%%%%%%%%%%%%%%%%%%%%%%%%%%%%%%%%%%%%%%%%%%%%
\begin{question}{diff11}
\luadirect{
  a=math.random(2, 9);
  b=math.random(2, 9);
  trig={'sin', 'cos'};
  t=math.random(1, 2);
  s=math.tointeger((-1)^(math.random(0,1)));
  g=a..'*x+'..s..'*'..b;
  f=trig[t]..'('..g..')';
  formula=execMaxima(f);
  correct1=execMaxima('diff('..f..', x)');
  wrong1=execMaxima('diff((-1)*'..f..', x)');
  wrong2=execMaxima('diff(2*'..f..', x)');
  wrong3=execMaxima('diff((-2)*'..f..', x)');
  wrong4=execMaxima('diff('..f..', x)/diff('..g..', x)');
}
函数 \(f(x)=\var{formula}\)の導函数\(f'(x)\)を求めなさい.
  \begin{choiceshoriz}
    \correctchoice{\(\var{correct1}\)}
    \wrongchoice{\(\var{wrong1}\)}
    \wrongchoice{\(\var{wrong2}\)}
    \wrongchoice{\(\var{wrong3}\)}
    \wrongchoice{\(\var{wrong4}\)}
  \end{choiceshoriz}
\end{question}
%%%%%%%%%%%%%%%%%%%%%%%%%%%%%%%%%%%%%%%%%%%%%%%%%%%%%%%%%%%%%
\begin{question}{diff12}
\luadirect{
  a=math.random(2, 9);
  b=math.random(2, 9);
  s=math.tointeger((-1)^(math.random(0,1)));
  g=a..'*x+'..s..'*'..b;
  f='tan('..g..')';
  formula=execMaxima(f);
  correct1=execMaxima('trigsimp(diff('..f..', x))');
  wrong1=execMaxima('trigsimp(diff((-1)*'..f..', x))');
  wrong2=execMaxima('trigsimp(diff(2*'..f..', x))');
  wrong3=execMaxima('trigsimp(diff((-2)*'..f..', x))');
  wrong4=execMaxima('trigsimp(diff('..f..', x)/diff('..g..', x))');
}
函数 \(f(x)=\var{formula}\)の導函数\(f'(x)\)を求めなさい.
  \begin{choiceshoriz}
    \correctchoice{\(\var{correct1}\)}
    \wrongchoice{\(\var{wrong1}\)}
    \wrongchoice{\(\var{wrong2}\)}
    \wrongchoice{\(\var{wrong3}\)}
    \wrongchoice{\(\var{wrong4}\)}
  \end{choiceshoriz}
\end{question}

%%%%%%%%%%%%%%%%%%%%%%%%%%%%%%%%%%%%%%%%%%%%%%%%%%%%%%%%%%%%%
\begin{question}{diff13}
\luadirect{
  a=math.random(1, 4);
  b=math.random(5, 9);
  g='sin('..a..'*x)';
  h='cos('..b..'*x)';
  f=g..'*'..h;
  formula=execMaxima(f);
  correct=execSage('diff('..f..', x)');
  wrong1=execSage('diff('..g..', x)*diff('..h..', x)');
  wrong2=execSage('diff('..g..', x)*'..h..'-'..g..'*diff('..h..', x)');
  wrong3=execSage('-diff('..g..', x)*'..h..'+'..g..'*diff('..h..', x)');
  wrong4=execMaxima('-diff('..g..', x)*diff('..h..', x)');
}
函数 \(f(x)=\var{formula}\)の導函数\(f'(x)\)を求めなさい.
  \begin{choices}
   \correctchoice{\(\var{correct}\)}
    \wrongchoice{\(\var{wrong1}\)}
    \wrongchoice{\(\var{wrong2}\)}
    \wrongchoice{\(\var{wrong3}\)}
    \wrongchoice{\(\var{wrong4}\)}
  \end{choices}
\end{question}

%%%%%%%%%%%%%%%%%%%%%%%%%%%%%%%%%%%%%%%%%%%%%%%%%%%%%%%%%%%%%
\begin{question}{diff14}
\luadirect{
  a=math.random(2, 5);
  b=math.random(2, 9);
  f='exp('..a..'*x+'..b..')';
  formula=execMaxima(f);
  correct=execMaxima('diff('..f..', x)');
  wrong1=execMaxima('('..a..'*x+'..b..')*exp('..a..'*x+'..b..'-1)');
  wrong2=execMaxima('('..a..'*x+'..b..')*exp('..a..'*x+'..b..')');
  wrong3=execMaxima(f);
}
函数 \(f(x)=\var{formula}\)の導函数\(f'(x)\)を求めなさい.
  \begin{choiceshoriz}
   \correctchoice{\(\var{correct}\)}
   \wrongchoice{\(\var{wrong1}\)}
   \wrongchoice{\(\var{wrong2}\)}
   \wrongchoice{\(\var{wrong3}\)}
  \end{choiceshoriz}
\end{question}

%%%%%%%%%%%%%%%%%%%%%%%%%%%%%%%%%%%%%%%%%%%%%%%%%%%%%%%%%%%%%
\begin{question}{diff15}
\luadirect{
  a=math.random(2, 5);
  b=math.random(6, 9);
  g=a..'*x+'..b;
  f='log('..g..')';
  formula=execMaxima(f);
  correct=execMaxima('diff('..f..', x)');
  wrong1=execMaxima('('..a..'*x+'..b..')*log('..a..'*x+'..b..'-1)');
  wrong2=execMaxima('1/('..g..')');
  wrong3=execMaxima(f);
}
函数 \(f(x)=\var{formula}\)の導函数\(f'(x)\)を求めなさい.
  \begin{choiceshoriz}
   \correctchoice{\(\var{correct}\)}
   \wrongchoice{\(\var{wrong1}\)}
   \wrongchoice{\(\var{wrong2}\)}
   \wrongchoice{\(\var{wrong3}\)}
  \end{choiceshoriz}
\end{question}

%%%%%%%%%%%%%%%%%%%%%%%%%%%%%%%%%%%%%%%%%%%%%%%%%%%%%%%%%%%%%
% 問終了 %%%%%%%%%%%%%%%%%%%%%%%%%%%%%%%%%%%%%%%%%%%%%%%%%%%%%
%%%%%%%%%%%%%%%%%%%%%%%%%%%%%%%%%%%%%%%%%%%%%%%%%%%%%%%%%%%%%
}
\end{document}



