\documentclass[a4paper]{ltjsarticle}
\usepackage{amsmath, amssymb}
\usepackage{tikz}
\usepackage{luacode}
\usepackage[box,completemulti,lang=JA]{automultiplechoice}
\newcommand*{\var}[1]{\luaexec{tex.print(#1)}}
\begin{document}
\AMCboxDimensions{shape=oval,width=1.8ex,height=2.5ex}
\luaexec{math.randomseed(190522)} % \onecopy{} の外に配置

\begin{luacode*}
function execSage(command)
  local sage_command="sage -c 'print(latex("..command.."))'"
  local handle=io.popen(sage_command, "r")
  local content=string.gsub(handle:read("*all"), "\n", "")
  handle:close()
  return content
end
function execMaxima(command)
  local maxima_command="echo 'tex1("..command..");' | maxima --very-quiet "
  local handle=io.popen(maxima_command, "r")
  local content=string.gsub(handle:read("*all"), "\n", "")
  handle:close()
  return content
end
\end{luacode*}

%%%%%%%%%%%%%%%%%%%%%%%%%%%%%%%%%%%%%%%%%%%%%%%%%%%%%%%%%%%%%
\element{sinp}{
 \begin{questionmult}{sinp01}
  方程式\(\sin x = 0\, (0 \leqq x \leqq 2\pi)\)の解を全て選択しなさい。
  \begin{choiceshoriz}
    \correctchoice{\(0\)}
    \wrongchoice{\(\frac{\pi}{6}\)}
    \wrongchoice{\(\frac{\pi}{4}\)}
    \wrongchoice{\(\frac{\pi}{3}\)}
    \wrongchoice{\(\frac{\pi}{2}\)}
    \wrongchoice{\(\frac{2}{3}\pi\)}
    \wrongchoice{\(\frac{3}{4}\pi\)}
    \wrongchoice{\(\frac{5}{6}\pi\)}
    \correctchoice{\(\pi\)}
    \wrongchoice{\(\frac{7}{6}\pi\)}
    \wrongchoice{\(\frac{5}{4}\pi\)}
    \wrongchoice{\(\frac{4}{3}\pi\)}
    \wrongchoice{\(\frac{3}{2}\pi\)}
    \wrongchoice{\(\frac{5}{3}\pi\)}
    \wrongchoice{\(\frac{7}{4}\pi\)}
    \wrongchoice{\(\frac{11}{6}\pi\)}
    \correctchoice{\(2\pi\)}
  \end{choiceshoriz}
 \end{questionmult}
}
%%%%%%%%%%%%%%%%%%%%%%%%%%%%%%%%%%%%%%%%%%%%%%%%%%%%%%%%%%%%%
\element{sinp}{
 \begin{questionmult}{sinp02}
  方程式\(\sin x = \frac{1}{2}\, (0 \leqq x \leqq 2\pi)\)の解を全て選択しなさい。
  \begin{choiceshoriz}
    \wrongchoice{\(0\)}
    \correctchoice{\(\frac{\pi}{6}\)}
    \wrongchoice{\(\frac{\pi}{4}\)}
    \wrongchoice{\(\frac{\pi}{3}\)}
    \wrongchoice{\(\frac{\pi}{2}\)}
    \wrongchoice{\(\frac{2}{3}\pi\)}
    \wrongchoice{\(\frac{3}{4}\pi\)}
    \correctchoice{\(\frac{5}{6}\pi\)}
    \wrongchoice{\(\pi\)}
    \wrongchoice{\(\frac{7}{6}\pi\)}
    \wrongchoice{\(\frac{5}{4}\pi\)}
    \wrongchoice{\(\frac{4}{3}\pi\)}
    \wrongchoice{\(\frac{3}{2}\pi\)}
    \wrongchoice{\(\frac{5}{3}\pi\)}
    \wrongchoice{\(\frac{7}{4}\pi\)}
    \wrongchoice{\(\frac{11}{6}\pi\)}
    \wrongchoice{\(2\pi\)}
  \end{choiceshoriz}
 \end{questionmult}
}
%%%%%%%%%%%%%%%%%%%%%%%%%%%%%%%%%%%%%%%%%%%%%%%%%%%%%%%%%%%%%
\element{sinp}{
 \begin{questionmult}{sinp03}
  方程式\(\sin x = \frac{1}{\sqrt{2}}\, (0 \leqq x \leqq 2\pi)\)の解を全て選択しなさい。
  \begin{choiceshoriz}
    \wrongchoice{\(0\)}
    \wrongchoice{\(\frac{\pi}{6}\)}
    \correctchoice{\(\frac{\pi}{4}\)}
    \wrongchoice{\(\frac{\pi}{3}\)}
    \wrongchoice{\(\frac{\pi}{2}\)}
    \wrongchoice{\(\frac{2}{3}\pi\)}
    \correctchoice{\(\frac{3}{4}\pi\)}
    \wrongchoice{\(\frac{5}{6}\pi\)}
    \wrongchoice{\(\pi\)}
    \wrongchoice{\(\frac{7}{6}\pi\)}
    \wrongchoice{\(\frac{5}{4}\pi\)}
    \wrongchoice{\(\frac{4}{3}\pi\)}
    \wrongchoice{\(\frac{3}{2}\pi\)}
    \wrongchoice{\(\frac{5}{3}\pi\)}
    \wrongchoice{\(\frac{7}{4}\pi\)}
    \wrongchoice{\(\frac{11}{6}\pi\)}
    \wrongchoice{\(2\pi\)}
  \end{choiceshoriz}
 \end{questionmult}
}
%%%%%%%%%%%%%%%%%%%%%%%%%%%%%%%%%%%%%%%%%%%%%%%%%%%%%%%%%%%%%
\element{sinp}{
 \begin{questionmult}{sinp04}
  方程式\(\sin x = \frac{\sqrt{3}}{2}\, (0 \leqq x \leqq 2\pi)\)の解を全て選択しなさい。
  \begin{choiceshoriz}
    \wrongchoice{\(0\)}
    \wrongchoice{\(\frac{\pi}{6}\)}
    \wrongchoice{\(\frac{\pi}{4}\)}
    \correctchoice{\(\frac{\pi}{3}\)}
    \wrongchoice{\(\frac{\pi}{2}\)}
    \correctchoice{\(\frac{2}{3}\pi\)}
    \wrongchoice{\(\frac{3}{4}\pi\)}
    \wrongchoice{\(\frac{5}{6}\pi\)}
    \wrongchoice{\(\pi\)}
    \wrongchoice{\(\frac{7}{6}\pi\)}
    \wrongchoice{\(\frac{5}{4}\pi\)}
    \wrongchoice{\(\frac{4}{3}\pi\)}
    \wrongchoice{\(\frac{3}{2}\pi\)}
    \wrongchoice{\(\frac{5}{3}\pi\)}
    \wrongchoice{\(\frac{7}{4}\pi\)}
    \wrongchoice{\(\frac{11}{6}\pi\)}
    \wrongchoice{\(2\pi\)}
  \end{choiceshoriz}
 \end{questionmult}
}
%%%%%%%%%%%%%%%%%%%%%%%%%%%%%%%%%%%%%%%%%%%%%%%%%%%%%%%%%%%%%
\element{sinp}{
 \begin{questionmult}{sinp05}
  方程式\(\sin x = 1\, (0 \leqq x \leqq 2\pi)\)の解を全て選択しなさい。
  \begin{choiceshoriz}
    \wrongchoice{\(0\)}
    \wrongchoice{\(\frac{\pi}{6}\)}
    \wrongchoice{\(\frac{\pi}{4}\)}
    \wrongchoice{\(\frac{\pi}{3}\)}
    \correctchoice{\(\frac{\pi}{2}\)}
    \wrongchoice{\(\frac{2}{3}\pi\)}
    \wrongchoice{\(\frac{3}{4}\pi\)}
    \wrongchoice{\(\frac{5}{6}\pi\)}
    \wrongchoice{\(\pi\)}
    \wrongchoice{\(\frac{7}{6}\pi\)}
    \wrongchoice{\(\frac{5}{4}\pi\)}
    \wrongchoice{\(\frac{4}{3}\pi\)}
    \wrongchoice{\(\frac{3}{2}\pi\)}
    \wrongchoice{\(\frac{5}{3}\pi\)}
    \wrongchoice{\(\frac{7}{4}\pi\)}
    \wrongchoice{\(\frac{11}{6}\pi\)}
    \wrongchoice{\(2\pi\)}
  \end{choiceshoriz}
 \end{questionmult}
}
%%%%%%%%%%%%%%%%%%%%%%%%%%%%%%%%%%%%%%%%%%%%%%%%%%%%%%%%%%%%%
\element{sinm}{
 \begin{questionmult}{sinm02}
  方程式\(\sin x = -\frac{1}{2}\, (0 \leqq x \leqq 2\pi)\)の解を全て選択しなさい。
  \begin{choiceshoriz}
    \wrongchoice{\(0\)}
    \wrongchoice{\(\frac{\pi}{6}\)}
    \wrongchoice{\(\frac{\pi}{4}\)}
    \wrongchoice{\(\frac{\pi}{3}\)}
    \wrongchoice{\(\frac{\pi}{2}\)}
    \wrongchoice{\(\frac{2}{3}\pi\)}
    \wrongchoice{\(\frac{3}{4}\pi\)}
    \wrongchoice{\(\frac{5}{6}\pi\)}
    \wrongchoice{\(\pi\)}
    \correctchoice{\(\frac{7}{6}\pi\)}
    \wrongchoice{\(\frac{5}{4}\pi\)}
    \wrongchoice{\(\frac{4}{3}\pi\)}
    \wrongchoice{\(\frac{3}{2}\pi\)}
    \wrongchoice{\(\frac{5}{3}\pi\)}
    \wrongchoice{\(\frac{7}{4}\pi\)}
    \correctchoice{\(\frac{11}{6}\pi\)}
    \wrongchoice{\(2\pi\)}
  \end{choiceshoriz}
 \end{questionmult}
}
%%%%%%%%%%%%%%%%%%%%%%%%%%%%%%%%%%%%%%%%%%%%%%%%%%%%%%%%%%%%%
\element{sinm}{
 \begin{questionmult}{sinm03}
  方程式\(\sin x = -\frac{1}{\sqrt{2}}\, (0 \leqq x \leqq 2\pi)\)の解を全て選択しなさい。
  \begin{choiceshoriz}
    \wrongchoice{\(0\)}
    \wrongchoice{\(\frac{\pi}{6}\)}
    \wrongchoice{\(\frac{\pi}{4}\)}
    \wrongchoice{\(\frac{\pi}{3}\)}
    \wrongchoice{\(\frac{\pi}{2}\)}
    \wrongchoice{\(\frac{2}{3}\pi\)}
    \wrongchoice{\(\frac{3}{4}\pi\)}
    \wrongchoice{\(\frac{5}{6}\pi\)}
    \wrongchoice{\(\pi\)}
    \wrongchoice{\(\frac{7}{6}\pi\)}
    \correctchoice{\(\frac{5}{4}\pi\)}
    \wrongchoice{\(\frac{4}{3}\pi\)}
    \wrongchoice{\(\frac{3}{2}\pi\)}
    \wrongchoice{\(\frac{5}{3}\pi\)}
    \correctchoice{\(\frac{7}{4}\pi\)}
    \wrongchoice{\(\frac{11}{6}\pi\)}
    \wrongchoice{\(2\pi\)}
  \end{choiceshoriz}
 \end{questionmult}
}
%%%%%%%%%%%%%%%%%%%%%%%%%%%%%%%%%%%%%%%%%%%%%%%%%%%%%%%%%%%%%
\element{sinm}{
 \begin{questionmult}{sinm04}
  方程式\(\sin x = -\frac{\sqrt{3}}{2}\, (0 \leqq x \leqq 2\pi)\)の解を全て選択しなさい。
  \begin{choiceshoriz}
    \wrongchoice{\(0\)}
    \wrongchoice{\(\frac{\pi}{6}\)}
    \wrongchoice{\(\frac{\pi}{4}\)}
    \wrongchoice{\(\frac{\pi}{3}\)}
    \wrongchoice{\(\frac{\pi}{2}\)}
    \wrongchoice{\(\frac{2}{3}\pi\)}
    \wrongchoice{\(\frac{3}{4}\pi\)}
    \wrongchoice{\(\frac{5}{6}\pi\)}
    \wrongchoice{\(\pi\)}
    \wrongchoice{\(\frac{7}{6}\pi\)}
    \wrongchoice{\(\frac{5}{4}\pi\)}
    \correctchoice{\(\frac{4}{3}\pi\)}
    \wrongchoice{\(\frac{3}{2}\pi\)}
    \correctchoice{\(\frac{5}{3}\pi\)}
    \wrongchoice{\(\frac{7}{4}\pi\)}
    \wrongchoice{\(\frac{11}{6}\pi\)}
    \wrongchoice{\(2\pi\)}
  \end{choiceshoriz}
 \end{questionmult}
}
%%%%%%%%%%%%%%%%%%%%%%%%%%%%%%%%%%%%%%%%%%%%%%%%%%%%%%%%%%%%%
\element{sinm}{
 \begin{questionmult}{sinm05}
  方程式\(\sin x = -1\, (0 \leqq x \leqq 2\pi)\)の解を全て選択しなさい。
  \begin{choiceshoriz}
    \wrongchoice{\(0\)}
    \wrongchoice{\(\frac{\pi}{6}\)}
    \wrongchoice{\(\frac{\pi}{4}\)}
    \wrongchoice{\(\frac{\pi}{3}\)}
    \wrongchoice{\(\frac{\pi}{2}\)}
    \wrongchoice{\(\frac{2}{3}\pi\)}
    \wrongchoice{\(\frac{3}{4}\pi\)}
    \wrongchoice{\(\frac{5}{6}\pi\)}
    \wrongchoice{\(\pi\)}
    \wrongchoice{\(\frac{7}{6}\pi\)}
    \wrongchoice{\(\frac{5}{4}\pi\)}
    \wrongchoice{\(\frac{4}{3}\pi\)}
    \correctchoice{\(\frac{3}{2}\pi\)}
    \wrongchoice{\(\frac{5}{3}\pi\)}
    \wrongchoice{\(\frac{7}{4}\pi\)}
    \wrongchoice{\(\frac{11}{6}\pi\)}
    \wrongchoice{\(2\pi\)}
  \end{choiceshoriz}
 \end{questionmult}
}
%%%%%%%%%%%%%%%%%%%%%%%%%%%%%%%%%%%%%%%%%%%%%%%%%%%%%%%%%%%%%
\element{cosp}{
 \begin{questionmult}{cosp01}
  方程式\(\cos x = 1\, (0 \leqq x \leqq 2\pi)\)の解を全て選択しなさい。
  \begin{choiceshoriz}
    \correctchoice{\(0\)}
    \wrongchoice{\(\frac{\pi}{6}\)}
    \wrongchoice{\(\frac{\pi}{4}\)}
    \wrongchoice{\(\frac{\pi}{3}\)}
    \wrongchoice{\(\frac{\pi}{2}\)}
    \wrongchoice{\(\frac{2}{3}\pi\)}
    \wrongchoice{\(\frac{3}{4}\pi\)}
    \wrongchoice{\(\frac{5}{6}\pi\)}
    \wrongchoice{\(\pi\)}
    \wrongchoice{\(\frac{7}{6}\pi\)}
    \wrongchoice{\(\frac{5}{4}\pi\)}
    \wrongchoice{\(\frac{4}{3}\pi\)}
    \wrongchoice{\(\frac{3}{2}\pi\)}
    \wrongchoice{\(\frac{5}{3}\pi\)}
    \wrongchoice{\(\frac{7}{4}\pi\)}
    \wrongchoice{\(\frac{11}{6}\pi\)}
    \correctchoice{\(2\pi\)}
  \end{choiceshoriz}
 \end{questionmult}
}
%%%%%%%%%%%%%%%%%%%%%%%%%%%%%%%%%%%%%%%%%%%%%%%%%%%%%%%%%%%%%
\element{cosp}{
 \begin{questionmult}{cosp02}
  方程式\(\cos x = \frac{\sqrt{3}}{2}\, (0 \leqq x \leqq 2\pi)\)の解を全て選択しなさい。
  \begin{choiceshoriz}
    \wrongchoice{\(0\)}
    \correctchoice{\(\frac{\pi}{6}\)}
    \wrongchoice{\(\frac{\pi}{4}\)}
    \wrongchoice{\(\frac{\pi}{3}\)}
    \wrongchoice{\(\frac{\pi}{2}\)}
    \wrongchoice{\(\frac{2}{3}\pi\)}
    \wrongchoice{\(\frac{3}{4}\pi\)}
    \wrongchoice{\(\frac{5}{6}\pi\)}
    \wrongchoice{\(\pi\)}
    \wrongchoice{\(\frac{7}{6}\pi\)}
    \wrongchoice{\(\frac{5}{4}\pi\)}
    \wrongchoice{\(\frac{4}{3}\pi\)}
    \wrongchoice{\(\frac{3}{2}\pi\)}
    \wrongchoice{\(\frac{5}{3}\pi\)}
    \wrongchoice{\(\frac{7}{4}\pi\)}
    \correctchoice{\(\frac{11}{6}\pi\)}
    \wrongchoice{\(2\pi\)}
  \end{choiceshoriz}
 \end{questionmult}
}
%%%%%%%%%%%%%%%%%%%%%%%%%%%%%%%%%%%%%%%%%%%%%%%%%%%%%%%%%%%%%
\element{cosp}{
 \begin{questionmult}{cosp03}
  方程式\(\cos x = \frac{1}{\sqrt{2}}\, (0 \leqq x \leqq 2\pi)\)の解を全て選択しなさい。
  \begin{choiceshoriz}
    \wrongchoice{\(0\)}
    \wrongchoice{\(\frac{\pi}{6}\)}
    \correctchoice{\(\frac{\pi}{4}\)}
    \wrongchoice{\(\frac{\pi}{3}\)}
    \wrongchoice{\(\frac{\pi}{2}\)}
    \wrongchoice{\(\frac{2}{3}\pi\)}
    \wrongchoice{\(\frac{3}{4}\pi\)}
    \wrongchoice{\(\frac{5}{6}\pi\)}
    \wrongchoice{\(\pi\)}
    \wrongchoice{\(\frac{7}{6}\pi\)}
    \wrongchoice{\(\frac{5}{4}\pi\)}
    \wrongchoice{\(\frac{4}{3}\pi\)}
    \wrongchoice{\(\frac{3}{2}\pi\)}
    \wrongchoice{\(\frac{5}{3}\pi\)}
    \correctchoice{\(\frac{7}{4}\pi\)}
    \wrongchoice{\(\frac{11}{6}\pi\)}
    \wrongchoice{\(2\pi\)}
  \end{choiceshoriz}
 \end{questionmult}
}
%%%%%%%%%%%%%%%%%%%%%%%%%%%%%%%%%%%%%%%%%%%%%%%%%%%%%%%%%%%%%
\element{cosp}{
 \begin{questionmult}{cosp04}
  方程式\(\cos x = \frac{1}{2}\, (0 \leqq x \leqq 2\pi)\)の解を全て選択しなさい。
  \begin{choiceshoriz}
    \wrongchoice{\(0\)}
    \wrongchoice{\(\frac{\pi}{6}\)}
    \wrongchoice{\(\frac{\pi}{4}\)}
    \correctchoice{\(\frac{\pi}{3}\)}
    \wrongchoice{\(\frac{\pi}{2}\)}
    \wrongchoice{\(\frac{2}{3}\pi\)}
    \wrongchoice{\(\frac{3}{4}\pi\)}
    \wrongchoice{\(\frac{5}{6}\pi\)}
    \wrongchoice{\(\pi\)}
    \wrongchoice{\(\frac{7}{6}\pi\)}
    \wrongchoice{\(\frac{5}{4}\pi\)}
    \wrongchoice{\(\frac{4}{3}\pi\)}
    \wrongchoice{\(\frac{3}{2}\pi\)}
    \correctchoice{\(\frac{5}{3}\pi\)}
    \wrongchoice{\(\frac{7}{4}\pi\)}
    \wrongchoice{\(\frac{11}{6}\pi\)}
    \wrongchoice{\(2\pi\)}
  \end{choiceshoriz}
 \end{questionmult}
}
%%%%%%%%%%%%%%%%%%%%%%%%%%%%%%%%%%%%%%%%%%%%%%%%%%%%%%%%%%%%%
\element{cosp}{
 \begin{questionmult}{cosp05}
  方程式\(\cos x = 0\, (0 \leqq x \leqq 2\pi)\)の解を全て選択しなさい。
  \begin{choiceshoriz}
    \wrongchoice{\(0\)}
    \wrongchoice{\(\frac{\pi}{6}\)}
    \wrongchoice{\(\frac{\pi}{4}\)}
    \wrongchoice{\(\frac{\pi}{3}\)}
    \correctchoice{\(\frac{\pi}{2}\)}
    \wrongchoice{\(\frac{2}{3}\pi\)}
    \wrongchoice{\(\frac{3}{4}\pi\)}
    \wrongchoice{\(\frac{5}{6}\pi\)}
    \wrongchoice{\(\pi\)}
    \wrongchoice{\(\frac{7}{6}\pi\)}
    \wrongchoice{\(\frac{5}{4}\pi\)}
    \wrongchoice{\(\frac{4}{3}\pi\)}
    \correctchoice{\(\frac{3}{2}\pi\)}
    \wrongchoice{\(\frac{5}{3}\pi\)}
    \wrongchoice{\(\frac{7}{4}\pi\)}
    \wrongchoice{\(\frac{11}{6}\pi\)}
    \wrongchoice{\(2\pi\)}
  \end{choiceshoriz}
 \end{questionmult}
}
%%%%%%%%%%%%%%%%%%%%%%%%%%%%%%%%%%%%%%%%%%%%%%%%%%%%%%%%%%%%%
\element{cosm}{
 \begin{questionmult}{cosm01}
  方程式\(\cos x = -\frac{1}{2}\, (0 \leqq x \leqq 2\pi)\)の解を全て選択しなさい。
  \begin{choiceshoriz}
    \wrongchoice{\(0\)}
    \wrongchoice{\(\frac{\pi}{6}\)}
    \wrongchoice{\(\frac{\pi}{4}\)}
    \wrongchoice{\(\frac{\pi}{3}\)}
    \wrongchoice{\(\frac{\pi}{2}\)}
    \correctchoice{\(\frac{2}{3}\pi\)}
    \wrongchoice{\(\frac{3}{4}\pi\)}
    \wrongchoice{\(\frac{5}{6}\pi\)}
    \wrongchoice{\(\pi\)}
    \wrongchoice{\(\frac{7}{6}\pi\)}
    \wrongchoice{\(\frac{5}{4}\pi\)}
    \correctchoice{\(\frac{4}{3}\pi\)}
    \wrongchoice{\(\frac{3}{2}\pi\)}
    \wrongchoice{\(\frac{5}{3}\pi\)}
    \wrongchoice{\(\frac{7}{4}\pi\)}
    \wrongchoice{\(\frac{11}{6}\pi\)}
    \wrongchoice{\(2\pi\)}
  \end{choiceshoriz}
 \end{questionmult}
}
%%%%%%%%%%%%%%%%%%%%%%%%%%%%%%%%%%%%%%%%%%%%%%%%%%%%%%%%%%%%%
\element{cosm}{
 \begin{questionmult}{cosm02}
  方程式\(\cos x = -\frac{1}{\sqrt{2}}\, (0 \leqq x \leqq 2\pi)\)の解を全て選択しなさい。
  \begin{choiceshoriz}
    \wrongchoice{\(0\)}
    \wrongchoice{\(\frac{\pi}{6}\)}
    \wrongchoice{\(\frac{\pi}{4}\)}
    \wrongchoice{\(\frac{\pi}{3}\)}
    \wrongchoice{\(\frac{\pi}{2}\)}
    \wrongchoice{\(\frac{2}{3}\pi\)}
    \correctchoice{\(\frac{3}{4}\pi\)}
    \wrongchoice{\(\frac{5}{6}\pi\)}
    \wrongchoice{\(\pi\)}
    \wrongchoice{\(\frac{7}{6}\pi\)}
    \correctchoice{\(\frac{5}{4}\pi\)}
    \wrongchoice{\(\frac{4}{3}\pi\)}
    \wrongchoice{\(\frac{3}{2}\pi\)}
    \wrongchoice{\(\frac{5}{3}\pi\)}
    \wrongchoice{\(\frac{7}{4}\pi\)}
    \wrongchoice{\(\frac{11}{6}\pi\)}
    \wrongchoice{\(2\pi\)}
  \end{choiceshoriz}
 \end{questionmult}
}
%%%%%%%%%%%%%%%%%%%%%%%%%%%%%%%%%%%%%%%%%%%%%%%%%%%%%%%%%%%%%
\element{cosm}{
 \begin{questionmult}{cosm03}
  方程式\(\cos x = -\frac{\sqrt{3}}{2}\, (0 \leqq x \leqq 2\pi)\)の解を全て選択しなさい。
  \begin{choiceshoriz}
    \wrongchoice{\(0\)}
    \wrongchoice{\(\frac{\pi}{6}\)}
    \wrongchoice{\(\frac{\pi}{4}\)}
    \wrongchoice{\(\frac{\pi}{3}\)}
    \wrongchoice{\(\frac{\pi}{2}\)}
    \wrongchoice{\(\frac{2}{3}\pi\)}
    \wrongchoice{\(\frac{3}{4}\pi\)}
    \correctchoice{\(\frac{5}{6}\pi\)}
    \wrongchoice{\(\pi\)}
    \correctchoice{\(\frac{7}{6}\pi\)}
    \wrongchoice{\(\frac{5}{4}\pi\)}
    \wrongchoice{\(\frac{4}{3}\pi\)}
    \wrongchoice{\(\frac{3}{2}\pi\)}
    \wrongchoice{\(\frac{5}{3}\pi\)}
    \wrongchoice{\(\frac{7}{4}\pi\)}
    \wrongchoice{\(\frac{11}{6}\pi\)}
    \wrongchoice{\(2\pi\)}
  \end{choiceshoriz}
 \end{questionmult}
}
\element{tan}{
\begin{questionmult}{tan01}
 方程式\(\tan x = 0, (0\leqq x \leqq 2\pi)\)の解を全て選択しなさい。
 \begin{choiceshoriz}
    \correctchoice{\(0\)}
    \wrongchoice{\(\frac{\pi}{6}\)}
    \wrongchoice{\(\frac{\pi}{4}\)}
    \wrongchoice{\(\frac{\pi}{3}\)}
    \wrongchoice{\(\frac{\pi}{2}\)}
    \wrongchoice{\(\frac{2}{3}\pi\)}
    \wrongchoice{\(\frac{3}{4}\pi\)}
    \wrongchoice{\(\frac{5}{6}\pi\)}
    \correctchoice{\(\pi\)}
    \wrongchoice{\(\frac{7}{6}\pi\)}
    \wrongchoice{\(\frac{5}{4}\pi\)}
    \wrongchoice{\(\frac{4}{3}\pi\)}
    \wrongchoice{\(\frac{3}{2}\pi\)}
    \wrongchoice{\(\frac{5}{3}\pi\)}
    \wrongchoice{\(\frac{7}{4}\pi\)}
    \wrongchoice{\(\frac{11}{6}\pi\)}
    \correctchoice{\(2\pi\)}
 \end{choiceshoriz}
\end{questionmult}
}
\element{tan}{
\begin{questionmult}{tan02}
 方程式\(\tan x = \frac{1}{\sqrt{3}}, (0\leqq x \leqq 2\pi)\)の解を全て選択しなさい。
 \begin{choiceshoriz}
    \wrongchoice{\(0\)}
    \correctchoice{\(\frac{\pi}{6}\)}
    \wrongchoice{\(\frac{\pi}{4}\)}
    \wrongchoice{\(\frac{\pi}{3}\)}
    \wrongchoice{\(\frac{\pi}{2}\)}
    \wrongchoice{\(\frac{2}{3}\pi\)}
    \wrongchoice{\(\frac{3}{4}\pi\)}
    \wrongchoice{\(\frac{5}{6}\pi\)}
    \wrongchoice{\(\pi\)}
    \correctchoice{\(\frac{7}{6}\pi\)}
    \wrongchoice{\(\frac{5}{4}\pi\)}
    \wrongchoice{\(\frac{4}{3}\pi\)}
    \wrongchoice{\(\frac{3}{2}\pi\)}
    \wrongchoice{\(\frac{5}{3}\pi\)}
    \wrongchoice{\(\frac{7}{4}\pi\)}
    \wrongchoice{\(\frac{11}{6}\pi\)}
    \wrongchoice{\(2\pi\)}
 \end{choiceshoriz}
\end{questionmult}
}
\element{tan}{
\begin{questionmult}{tan03}
 方程式\(\tan x = 1, (0\leqq x \leqq 2\pi)\)の解を全て選択しなさい。
 \begin{choiceshoriz}
    \wrongchoice{\(0\)}
    \wrongchoice{\(\frac{\pi}{6}\)}
    \correctchoice{\(\frac{\pi}{4}\)}
    \wrongchoice{\(\frac{\pi}{3}\)}
    \wrongchoice{\(\frac{\pi}{2}\)}
    \wrongchoice{\(\frac{2}{3}\pi\)}
    \wrongchoice{\(\frac{3}{4}\pi\)}
    \wrongchoice{\(\frac{5}{6}\pi\)}
    \wrongchoice{\(\pi\)}
    \wrongchoice{\(\frac{7}{6}\pi\)}
    \correctchoice{\(\frac{5}{4}\pi\)}
    \wrongchoice{\(\frac{4}{3}\pi\)}
    \wrongchoice{\(\frac{3}{2}\pi\)}
    \wrongchoice{\(\frac{5}{3}\pi\)}
    \wrongchoice{\(\frac{7}{4}\pi\)}
    \wrongchoice{\(\frac{11}{6}\pi\)}
    \wrongchoice{\(2\pi\)}
 \end{choiceshoriz}
\end{questionmult}
}
\element{tan}{
\begin{questionmult}{tan04}
 方程式\(\tan x = \sqrt{3}, (0\leqq x \leqq 2\pi)\)の解を全て選択しなさい。
 \begin{choiceshoriz}
    \wrongchoice{\(0\)}
    \wrongchoice{\(\frac{\pi}{6}\)}
    \wrongchoice{\(\frac{\pi}{4}\)}
    \correctchoice{\(\frac{\pi}{3}\)}
    \wrongchoice{\(\frac{\pi}{2}\)}
    \wrongchoice{\(\frac{2}{3}\pi\)}
    \wrongchoice{\(\frac{3}{4}\pi\)}
    \wrongchoice{\(\frac{5}{6}\pi\)}
    \wrongchoice{\(\pi\)}
    \wrongchoice{\(\frac{7}{6}\pi\)}
    \wrongchoice{\(\frac{5}{4}\pi\)}
    \correctchoice{\(\frac{4}{3}\pi\)}
    \wrongchoice{\(\frac{3}{2}\pi\)}
    \wrongchoice{\(\frac{5}{3}\pi\)}
    \wrongchoice{\(\frac{7}{4}\pi\)}
    \wrongchoice{\(\frac{11}{6}\pi\)}
    \wrongchoice{\(2\pi\)}
 \end{choiceshoriz}
\end{questionmult}
}
\element{tan}{
\begin{questionmult}{tan05}
 方程式\(\tan x = -\sqrt{3}, (0\leqq x \leqq 2\pi)\)の解を全て選択しなさい。
 \begin{choiceshoriz}
    \wrongchoice{\(0\)}
    \wrongchoice{\(\frac{\pi}{6}\)}
    \wrongchoice{\(\frac{\pi}{4}\)}
    \wrongchoice{\(\frac{\pi}{3}\)}
    \wrongchoice{\(\frac{\pi}{2}\)}
    \correctchoice{\(\frac{2}{3}\pi\)}
    \wrongchoice{\(\frac{3}{4}\pi\)}
    \wrongchoice{\(\frac{5}{6}\pi\)}
    \wrongchoice{\(\pi\)}
    \wrongchoice{\(\frac{7}{6}\pi\)}
    \wrongchoice{\(\frac{5}{4}\pi\)}
    \wrongchoice{\(\frac{4}{3}\pi\)}
    \wrongchoice{\(\frac{3}{2}\pi\)}
    \correctchoice{\(\frac{5}{3}\pi\)}
    \wrongchoice{\(\frac{7}{4}\pi\)}
    \wrongchoice{\(\frac{11}{6}\pi\)}
    \wrongchoice{\(2\pi\)}
 \end{choiceshoriz}
\end{questionmult}
}
\element{tan}{
\begin{questionmult}{tan06}
 方程式\(\tan x = -1, (0\leqq x \leqq 2\pi)\)の解を全て選択しなさい。
 \begin{choiceshoriz}
    \wrongchoice{\(0\)}
    \wrongchoice{\(\frac{\pi}{6}\)}
    \wrongchoice{\(\frac{\pi}{4}\)}
    \wrongchoice{\(\frac{\pi}{3}\)}
    \wrongchoice{\(\frac{\pi}{2}\)}
    \wrongchoice{\(\frac{2}{3}\pi\)}
    \correctchoice{\(\frac{3}{4}\pi\)}
    \wrongchoice{\(\frac{5}{6}\pi\)}
    \wrongchoice{\(\pi\)}
    \wrongchoice{\(\frac{7}{6}\pi\)}
    \wrongchoice{\(\frac{5}{4}\pi\)}
    \wrongchoice{\(\frac{4}{3}\pi\)}
    \wrongchoice{\(\frac{3}{2}\pi\)}
    \wrongchoice{\(\frac{5}{3}\pi\)}
    \correctchoice{\(\frac{7}{4}\pi\)}
    \wrongchoice{\(\frac{11}{6}\pi\)}
    \wrongchoice{\(2\pi\)}
 \end{choiceshoriz}
\end{questionmult}
}
\element{tan}{
\begin{questionmult}{tan07}
 方程式\(\tan x = -\frac{1}{\sqrt{3}}, (0\leqq x \leqq 2\pi)\)の解を全て選択しなさい。
 \begin{choiceshoriz}
    \wrongchoice{\(0\)}
    \wrongchoice{\(\frac{\pi}{6}\)}
    \wrongchoice{\(\frac{\pi}{4}\)}
    \wrongchoice{\(\frac{\pi}{3}\)}
    \wrongchoice{\(\frac{\pi}{2}\)}
    \wrongchoice{\(\frac{2}{3}\pi\)}
    \wrongchoice{\(\frac{3}{4}\pi\)}
    \correctchoice{\(\frac{5}{6}\pi\)}
    \wrongchoice{\(\pi\)}
    \wrongchoice{\(\frac{7}{6}\pi\)}
    \wrongchoice{\(\frac{5}{4}\pi\)}
    \wrongchoice{\(\frac{4}{3}\pi\)}
    \wrongchoice{\(\frac{3}{2}\pi\)}
    \wrongchoice{\(\frac{5}{3}\pi\)}
    \wrongchoice{\(\frac{7}{4}\pi\)}
    \correctchoice{\(\frac{11}{6}\pi\)}
    \wrongchoice{\(2\pi\)}
 \end{choiceshoriz}
\end{questionmult}
}

\element{itrig}{
\begin{question}{itrig01}
\luaexec{
  table={'-2*\%pi', '-11*\%pi/6', '-7*\%pi/4', '-5*\%pi/3', '-3*\%pi/2', '-4*\%pi/3', '-5*\%pi/4', '-7*\%pi/6', '-\%pi', '-5*\%pi/6', '-3*\%pi/4', '-2*\%pi/3', '-\%pi/2', '-\%pi/3', '-\%pi/4', '-\%pi/6', '0', '\%pi/6', '\%pi/4', '\%pi/3', '\%pi/2', '2*\%pi/3', '3*\%pi/4', '5*\%pi/6', '\%pi', '7*\%pi/6', '5*\%pi/4', '4*\%pi/3', '3*\%pi/2', '5*\%pi/3', '7*\%pi/4', '11*\%pi/6', '2*\%pi'}
  i=math.random(13, 21);
  angle=table[i];
  y=execMaxima('sin('..angle..')');
  correct=execMaxima(angle);
  wrong={}
  if i>16 then for k=0, 7 do wrong[k+1]=execMaxima(table[math.fmod(i+k, 33)+1]) end;
  else for k=0, 7 do wrong[k+1]=execMaxima(table[math.fmod(i-k, 33)-1]) end; end;
}
$\arcsin(x)$は逆正弦函数とする.
$\arcsin\left(\var{y}\right)$の主値を求めなさい.
  \begin{choiceshoriz}
    \correctchoice{\(\var{correct}\)}
    \wrongchoice{\(\var{wrong[1]}\)}
    \wrongchoice{\(\var{wrong[2]}\)}
    \wrongchoice{\(\var{wrong[3]}\)}
    \wrongchoice{\(\var{wrong[4]}\)}
    \wrongchoice{\(\var{wrong[5]}\)}
    \wrongchoice{\(\var{wrong[6]}\)}
    \wrongchoice{\(\var{wrong[7]}\)}
    \wrongchoice{\(\var{wrong[8]}\)}
  \end{choiceshoriz}
\end{question}
}
\element{itrig}{
%%%%%%%%%%%%%%%%%%%%%%%%%%%%%%%%%%%%%%%%%%%%%%%%%%%%%%%%%%%%%
\begin{question}{itrig02}
\luaexec{
  table={'-2*\%pi', '-11*\%pi/6', '-7*\%pi/4', '-5*\%pi/3', '-3*\%pi/2', '-4*\%pi/3', '-5*\%pi/4', '-7*\%pi/6', '-\%pi', '-5*\%pi/6', '-3*\%pi/4', '-2*\%pi/3', '-\%pi/2', '-\%pi/3', '-\%pi/4', '-\%pi/6', '0', '\%pi/6', '\%pi/4', '\%pi/3', '\%pi/2', '2*\%pi/3', '3*\%pi/4', '5*\%pi/6', '\%pi', '7*\%pi/6', '5*\%pi/4', '4*\%pi/3', '3*\%pi/2', '5*\%pi/3', '7*\%pi/4', '11*\%pi/6', '2*\%pi'}
  i=math.random(17, 25); % 0 <= x <= pi
  angle=table[i];
  y=execMaxima('cos('..angle..')');
  correct=execMaxima(angle);
  wrong={}
  if i>20 then for k=0, 7 do wrong[k+1]=execMaxima(table[math.fmod(i+k, 33)+1]) end;
  else for k=0, 7 do wrong[k+1]=execMaxima(table[math.fmod(i-k, 33)-1]) end; end;
}
$\arccos(x)$は逆余弦函数とする.
$\arccos\left(\var{y}\right)$の主値を求めなさい.
  \begin{choiceshoriz}
    \correctchoice{\(\var{correct}\)}
    \wrongchoice{\(\var{wrong[1]}\)}
    \wrongchoice{\(\var{wrong[2]}\)}
    \wrongchoice{\(\var{wrong[3]}\)}
    \wrongchoice{\(\var{wrong[4]}\)}
    \wrongchoice{\(\var{wrong[5]}\)}
    \wrongchoice{\(\var{wrong[6]}\)}
    \wrongchoice{\(\var{wrong[7]}\)}
    \wrongchoice{\(\var{wrong[8]}\)}
  \end{choiceshoriz}
\end{question}
}

\onecopy{2}{

%%% ヘッダー開始: %%%   
\noindent{\bf 応用数学 演習07 \hfill 2019年5月22日}

\vspace{3ex}

{\small
  \setlength{\parindent}{0pt}\hspace*{\fill}\AMCcodeGridInt{学生番号}{8}\hspace*{\fill}
\begin{minipage}[b]{6.5cm}$\longleftarrow{}$\hspace{0ptplus1cm}
学生番号を左にマークし、下に氏名を記入してください。

\vspace{3ex}

\hfill\namefield{
  \fbox{
    \begin{minipage}{.9\linewidth}
      氏名\vspace*{.5cm}%\dotfill\vspace*{.5cm}\dotfill
      \vspace*{1mm}
    \end{minipage}
  }
}
\hfill\vspace{5ex}
\end{minipage}
\hspace*{\fill}
}

%\vspace{1ex}
%計算は自分のノートに書くこと.答案用紙に書いてはいけません.
%\multiSymbole{}の記号のある設問の正解は1個とは限りません。 0個の場合や複数の場合があります。
\hrulefill
\vspace{1ex}

%%% ヘッダー終了 %%%
%%%%%%%%%%%%%%%%%%%%%%%%%%%%%%%%%%%%%%%%%%%%%%%%%%%%%%%%%%%%%
\begin{question}{log03}
  \luaexec{
    a=math.random(2, 5);
    b=math.random(2, 10);
    c=math.random(2, 4);
    d=math.random(1, 8);
    e=a^c;
    table={-3, -2, -1, 1, 2, 3};
    r=math.random(1, 3);
    correct=(b-d)/2;
    wrong1=correct+table[r];
    wrong2=correct+table[r+1];
    wrong3=correct+table[r+2];
    wrong4=correct+table[r+3];
  }
  方程式\(\log_{\var{a}}(\var{b}-x)=\var{c}\log_{\var{e}}(x+\var{d})\)の解を求めよ.
  \begin{choiceshoriz}
    \correctchoice{\(\var{correct}\)}
    \wrongchoice{\(\var{wrong1}\)}
    \wrongchoice{\(\var{wrong2}\)}

    \wrongchoice{\(\var{wrong3}\)}
    \wrongchoice{\(\var{wrong4}\)}
  \end{choiceshoriz}   
\end{question}
%%%%%%%%%%%%%%%%%%%%%%%%%%%%%%%%%%%%%%%%%%%%%%%%%%%%%%%%%%%%%
% 三角函数
\cleargroup{all}
\copygroup{sinp}{all}
\copygroup{sinm}{all}
\copygroup{cosp}{all}
\copygroup{cosm}{all}
\copygroup{tan}{all}
\shufflegroup{all}
\insertgroup[1]{all}

%%%%%%%%%%%%%%%%%%%%%%%%%%%%%%%%%%%%%%%%%%%%%%%%%%%%%%%%%%%%%
% 導函数
%%%%%%%%%%%%%%%%%%%%%%%%%%%%%%%%%%%%%%%%%%%%%%%%%%%%%%%%%%%%%
\begin{question}{diff07}
\luaexec{
    array1={2, 4, 8};
    array2={3, 5, 7, 11};
    i=math.random(1, 3);
    j=math.random(1, 4);
    k=math.random(1, 4);
    l=math.random(1, 3);
    a=array1[i];
    b=array2[j];
    c=array2[k];
    d=array1[l];
}
函数\(f(x)=\frac{\var{a}x+\var{b}}{\var{c}x+\var{d}}\)の導函数\(f'(x)\)を求めなさい。

  \begin{choiceshoriz}
    \correctchoice{\(\frac{\var{a*d-b*c}}{(\var{c}x+\var{d})^{2}}\)}
    \wrongchoice{\(\frac{\var{a}}{\var{c}x+\var{d}}\)}
    \wrongchoice{\(\frac{\var{a}}{(\var{c}x+\var{d})^{2}}\)}
    \wrongchoice{\(\frac{\var{(a+1)*d-b*c}}{\var{c}x+\var{d}}\)}
    \wrongchoice{\(\frac{\var{a*d-b*c}}{\var{c}x+\var{d}}\)}
  \end{choiceshoriz}
\end{question}

%%%%%%%%%%%%%%%%%%%%%%%%%%%%%%%%%%%%%%%%%%%%%%%%%%%%%%%%%%%%%
\begin{question}{diff08}
\luaexec{
  a=math.random(2, 9);
  b=math.random(2, 9);
  c=math.random(2, 9);
  d=math.random(5, 12);
  g=a..'*x^2+'..b..'*x+'..c;
  f='('..a..'*x^2+'..b..'*x+'..c..')^'..d;
  formula=execMaxima(f);
  correct=execMaxima('diff('..f..', x)');
  wrong1=execMaxima(d..'*('..g..')^('..d..'-1)');
  wrong2=execMaxima(a..'*'..d..'*('..g..')^('..d..'-1)');
  wrong3=execMaxima(d..'*('..a..'*x+'..b..')*('..g..')^('..d..'-1)');
  wrong4=execMaxima(d..'*(diff('..g..', x)+'..c..')*('..g..')^('..d..'-1)');
}
函数\(f(x)=\var{formula}\)の導函数\(f'(x)\)を求めなさい。
  \begin{choiceshoriz}
    \correctchoice{\(\var{correct}\)}
    \wrongchoice{\(\var{wrong1}\)}
    \wrongchoice{\(\var{wrong2}\)}
    \wrongchoice{\(\var{wrong3}\)}
    \wrongchoice{\(\var{wrong4}\)}
  \end{choiceshoriz}
\end{question}

%%%%%%%%%%%%%%%%%%%%%%%%%%%%%%%%%%%%%%%%%%%%%%%%%%%%%%%%%%%%%
\begin{question}{diff11}
\luaexec{
  a=math.random(2, 9);
  b=math.random(2, 9);
  trig={'sin', 'cos'};
  t=math.random(1, 2);
  s=(-1)^(math.random(0,1));
  g=a..'*x+'..s..'*'..b;
  f=trig[t]..'('..g..')';
  formula=execMaxima(f);
  correct1=execMaxima('diff('..f..', x)');
  wrong1=execMaxima('diff((-1)*'..f..', x)');
  wrong2=execMaxima('diff(2*'..f..', x)');
  wrong3=execMaxima('diff((-2)*'..f..', x)');
  wrong4=execMaxima('diff('..f..', x)/diff('..g..', x)');
}
函数 \(f(x)=\var{formula}\)の導函数\(f'(x)\)を求めなさい.
  \begin{choiceshoriz}
    \correctchoice{\(\var{correct1}\)}
    \wrongchoice{\(\var{wrong1}\)}
    \wrongchoice{\(\var{wrong2}\)}
    \wrongchoice{\(\var{wrong3}\)}
    \wrongchoice{\(\var{wrong4}\)}
  \end{choiceshoriz}
\end{question}

%%%%%%%%%%%%%%%%%%%%%%%%%%%%%%%%%%%%%%%%%%%%%%%%%%%%%%%%%%%%%
\begin{question}{diff14}
\luaexec{
  a=math.random(2, 5);
  b=math.random(2, 9);
  b=math.random(1, 3);
  f='exp('..a..'*x+'..b..')';
  formula=execMaxima(f);
  correct=execMaxima('diff('..f..', x)');
  wrong1=execMaxima('('..a..'*x+'..b..')*exp('..a..'*x+'..b..'-1)');
  wrong2=execMaxima('('..a..'*x+'..b..')*exp('..a..'*x+'..b..')');
  wrong3=execMaxima(f);
}
函数 \(f(x)=\var{formula}\)の導函数\(f'(x)\)を求めなさい.
  \begin{choiceshoriz}
   \correctchoice{\(\var{correct}\)}
   \wrongchoice{\(\var{wrong1}\)}
   \wrongchoice{\(\var{wrong2}\)}
   \wrongchoice{\(\var{wrong3}\)}
  \end{choiceshoriz}
\end{question}

%%%%%%%%%%%%%%%%%%%%%%%%%%%%%%%%%%%%%%%%%%%%%%%%%%%%%%%%%%%%%
\shufflegroup{itrig}
\insertgroup[1]{itrig}

} % end of \onecopy{}
%%%%%%%%%%%%%%%%%%%%%%%%%%%%%%%%%%%%%%%%%%%%%%%%%%%%%%%%%%%%%
\end{document}