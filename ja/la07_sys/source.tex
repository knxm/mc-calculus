\documentclass[a4paper]{ltjsarticle}
\usepackage{amsmath, amssymb}
\usepackage{tikz, luacode}
\usepackage[box,completemulti,lang=JA]{automultiplechoice}
\newcommand*{\var}[1]{\luaexec{tex.print(#1)}}
\begin{document}
\AMCboxDimensions{shape=oval,width=1.8ex,height=2.5ex}
\AMCcodeVspace=0em
%\luaexec{math.randomseed(190606)} % \onecopy{} の外に配置

%\begin{luacode*}
%function execMaxima(command)
%  local maxima_command="echo 'tex1("..command..");' | maxima --very-quiet "
%  local handle=io.popen(maxima_command, "r")
%  local content=string.gsub(handle:read("*all"), "\n", "")
%  handle:close()
%  return content
%end
%function execSage(command)
%  local sage_command="sage -c 'print(latex("..command.."))'"
%  local handle=io.popen(sage_command, "r")
%  local content=string.gsub(handle:read("*all"), "\n", "")
%  handle:close()
%  return content
%end
%\end{luacode*}

%%%%%%%%%%%%%%%%%
\element{lasystem}{
\begin{question}{lasystem01}
以下の連立方程式を行列の基本変形によって解きなさい.
\begin{equation*}
\begin{cases}
 x + y - z =3\\
-3x -2y + 3z = -9\\
-3x -y + 2z = -8
\end{cases}
\end{equation*}
%\begin{equation*}
%\begin{pmatrix}
% 1 & 1 & -1 & 3\\
% - 3 & - 2 & 3 & - 9\\
% - 3 & -1 & 2 & - 8
%\end{pmatrix}
%\end{equation*}
\AMCOpen{lines=8, dots=false, framerule=0pt}{\wrongchoice[W]{誤}\scoring{0}\wrongchoice[P]{部}\scoring{1}\correctchoice[C]{正}\scoring{2}}
\end{question}
}

%%%%%%%%%%%%%%%%%
\element{lasystem}{
\begin{question}{lasystem02}
以下の連立方程式を行列の基本変形によって解きなさい.
\begin{equation*}
 \begin{cases}
 -x + y - z = 1\\
8x - 7y + 6z = -3\\
-5x + 5y -6z = 7
\end{cases}
\end{equation*}
%\begin{equation*}
% \begin{pmatrix}
% -1 & 1 & -1 & 1\\
% 8 & - 7 & 6 & - 3\\
% - 5 & 5 & - 6 & 7
%\end{pmatrix}
%\end{equation*}
\AMCOpen{lines=8, dots=false, framerule=0pt}{\wrongchoice[W]{誤}\scoring{0}\wrongchoice[P]{部}\scoring{1}\correctchoice[C]{正}\scoring{2}}
\end{question}
}

%%%%%%%%%%%%%%%%%
\element{lasystem}{
\begin{question}{lasystem03}
以下の連立方程式を行列の基本変形によって解きなさい.
\begin{equation*}
 \begin{cases}
 x - 2y - z = -2\\
 -2x + 4y + z = 2\\
 -3x + 5y + 3z = 7
\end{cases}
\end{equation*}
%\begin{equation*}
%\begin{pmatrix}
% 1 & - 2 & -1 & - 2\\
% - 2 & 4 & 1 & 2\\
% - 3 & 5 & 3 & 7
%\end{pmatrix}
%\end{equation*}
\AMCOpen{lines=8, dots=false, framerule=0pt}{\wrongchoice[W]{誤}\scoring{0}\wrongchoice[P]{部}\scoring{1}\correctchoice[C]{正}\scoring{2}}
\end{question}
}

%%%%%%%%%%%%%%%%%
\element{lasystem}{
\begin{question}{lasystem04}
以下の連立方程式を行列の基本変形によって解きなさい.
\begin{equation*}
 \begin{cases}
x-y+2z=-6\\
-x+3y-5z=9\\
2x+y-z=-7
\end{cases}
\end{equation*}
%\begin{equation*}
%\begin{pmatrix}
% 1 & -1 & 2 & -6\\
% -1 & 3 & -5 & 9\\
% 2 & 1 & -1 & -7
%\end{pmatrix}
%\end{equation*}
\AMCOpen{lines=8, dots=false, framerule=0pt}{\wrongchoice[W]{誤}\scoring{0}\wrongchoice[P]{部}\scoring{1}\correctchoice[C]{正}\scoring{2}}
\end{question}
}

%%%%%%%%%%%%%%%%%
\element{lasystem}{
\begin{question}{lasystem05}
以下の連立方程式を行列の基本変形によって解きなさい.
\begin{equation*}
 \begin{cases}
-7x-y-2z=5\\
-6x-y-2z=5\\
4x+y+3z=-7
\end{cases}
\end{equation*}
%\begin{equation*}
%\begin{pmatrix}
% - 7 & -1 & - 2 & 5\\
% - 6 & -1 & - 2 & 5\\
% 4 & 1 & 3 & - 7
%\end{pmatrix}
%\end{equation*}
\AMCOpen{lines=8, dots=false, framerule=0pt}{\wrongchoice[W]{誤}\scoring{0}\wrongchoice[P]{部}\scoring{1}\correctchoice[C]{正}\scoring{2}}
\end{question}
}

%%%%%%%%%%%%%%%%%
\element{lasystem}{
\begin{question}{lasystem06}
以下の連立方程式を行列の基本変形によって解きなさい.
\begin{equation*}
\begin{cases}
x+6y+3z=5\\
-2x-9y-5z=-8\\
-3x+y-3z=-3
\end{cases}
\end{equation*}
%\begin{equation*}
% \begin{cases}
%  1 & 6 & 3 & 5\\
%  -2 & -9 & -5 & -8\\
%  -3 & 1 & -3 & -3
% \end{cases}
%\end{equation*}
\AMCOpen{lines=8, dots=false, framerule=0pt}{\wrongchoice[W]{誤}\scoring{0}\wrongchoice[P]{部}\scoring{1}\correctchoice[C]{正}\scoring{2}}
\end{question}
}
\onecopy{2}{
%%% ヘッダー開始: %%%   
\noindent{\textbf{線形代数 演習07 \hfill 2019年6月6日}}

\vspace{3ex}

{\small
  \setlength{\parindent}{0pt}\hspace*{\fill}\AMCcodeGridInt{学生番号}{8}\hspace*{\fill}
\begin{minipage}[b]{6.5cm}
$\longleftarrow{}$\hspace{0ptplus1cm}
学生番号を左にマークし、下に氏名とNU-AppsGのメールアドレス(brxxyyzzz)を記入してください。
xxは名前から,yyは入学年度,zzzは学生番号下3桁です.@以降は必要ありません.

\vspace{3ex}

\hfill\namefield{
  \fbox{
    \begin{minipage}{.5\linewidth}
      氏名\vspace*{.5cm}%\dotfill\vspace*{.5cm}\dotfill
      \vspace*{1mm}
    \end{minipage}
  }
  \fbox{
    \begin{minipage}{.5\linewidth}
      NU-AppsG\vspace*{.5cm}%\dotfill\vspace*{.5cm}\dotfill
      \vspace*{1mm}
    \end{minipage}
 }
}
\hfill\vspace{5ex}
\end{minipage}
\hspace*{\fill}
}
% \multiSymbole{}の記号のある設問の正解は1個とは限りません。 0個の場合や複数の場合があります。

解答欄上部の「誤部正」は採点欄ですので,決して記入しないでください.

\vspace{1ex}
\hrulefill
\vspace{1ex}

%%% ヘッダー終了 %%%
%%%%%%%%%%%%%%%%%%%%%%%%%%%%%%%%%%%%%%%%%%%%%%%%%%%%%%%%%%%%%
\copygroup{lasystem}{all}
\shufflegroup{all}
\insertgroup[1]{all}

} % End of \onecopy{}
%%%%%%%%%%%%%%%%%%%%%%%%%%%%%%%%%%%%%%%%%%%%%%%%%%%%%%%%%%%%%
\end{document}
