\documentclass{ltjsarticle}
\usepackage{amsmath, amssymb}
\begin{document}
%%%%%%%%%%%%%%%
\begin{enumerate}
 \item 
\begin{align*}
\begin{pmatrix}
 - 7 & -1 & - 2 & 5\\
 - 6 & -1 & - 2 & 5\\
 4 & 1 & 3 & - 7
\end{pmatrix}\allowdisplaybreaks\\
&\ \ \ \xrightarrow{ \text{line} 2 \ -=\  \text{line} 1 } %
\begin{pmatrix}
 - 7 & -1 & - 2 & 5\\
 1 & 0 & 0 & 0\\
 4 & 1 & 3 & - 7
\end{pmatrix}\ \ \ \xrightarrow{ \text{line} 1 \ \leftrightarrow\  \text{line} 2 } %
\begin{pmatrix}
 1 & 0 & 0 & 0\\
 - 7 & -1 & - 2 & 5\\
 4 & 1 & 3 & - 7
\end{pmatrix}\allowdisplaybreaks\\
&\ \ \ \xrightarrow{ \text{line} 2 \ +=\  \text{line} 1 \times\left( 7 \right)} %
\begin{pmatrix}
 1 & 0 & 0 & 0\\
 0 & -1 & - 2 & 5\\
 4 & 1 & 3 & - 7
\end{pmatrix}
\ \ \ \xrightarrow{ \text{line} 3 \ -=\  \text{line} 1 \times\left( 4 \right)} %
\begin{pmatrix}
 1 & 0 & 0 & 0\\
 0 & -1 & - 2 & 5\\
 0 & 1 & 3 & - 7
\end{pmatrix}\allowdisplaybreaks\\
&\ \ \ \xrightarrow{ \text{line} 2 \ \leftrightarrow\  \text{line} 3 } %
\begin{pmatrix}
 1 & 0 & 0 & 0\\
 0 & 1 & 3 & - 7\\
 0 & -1 & - 2 & 5
\end{pmatrix}\ \ \ \xrightarrow{ \text{line} 3 \ +=\  \text{line} 2 } %
\begin{pmatrix}
 1 & 0 & 0 & 0\\
 0 & 1 & 3 & - 7\\
 0 & 0 & 1 & - 2
\end{pmatrix}\allowdisplaybreaks\\
&\ \ \ \xrightarrow{ \text{line} 2 \ -=\  \text{line} 3 \times\left( 3 \right)} %
\begin{pmatrix}
 1 & 0 & 0 & 0\\
 0 & 1 & 0 & -1\\
 0 & 0 & 1 & - 2
\end{pmatrix}
\end{align*}
したがって,$(x, y, z)=(0, -1, -2)$
%%%%%%%%%%%%%%%
\item
\begin{align*}
\begin{pmatrix}
 1 & -1 & 2 & - 6\\
 -1 & 3 & - 5 & 9\\
 2 & 1 & -1 & - 7
\end{pmatrix}\allowdisplaybreaks\\
&\ \ \ \xrightarrow{ \text{line} 2 \ +=\  \text{line} 1 } %
\begin{pmatrix}
 1 & -1 & 2 & - 6\\
 0 & 2 & - 3 & 3\\
 2 & 1 & -1 & - 7
\end{pmatrix}\ \ \ \xrightarrow{ \text{line} 3 \ -=\  \text{line} 1 \times\left( 2 \right)} %
\begin{pmatrix}
 1 & -1 & 2 & - 6\\
 0 & 2 & - 3 & 3\\
 0 & 3 & - 5 & 5
\end{pmatrix}\allowdisplaybreaks\\
 &\ \ \ \xrightarrow{ \text{line} 3 \ -=\  \text{line} 2 } %
\begin{pmatrix}
 1 & -1 & 2 & - 6\\
 0 & 2 & - 3 & 3\\
 0 & 1 & - 2 & 2
\end{pmatrix}\ \ \ \xrightarrow{ \text{line} 2 \ \leftrightarrow\  \text{line} 3 } %
\begin{pmatrix}
 1 & -1 & 2 & - 6\\
 0 & 1 & - 2 & 2\\
 0 & 2 & - 3 & 3
\end{pmatrix}\allowdisplaybreaks\\
 &\ \ \ \xrightarrow{ \text{line} 1 \ +=\  \text{line} 2 } %
\begin{pmatrix}
 1 & 0 & 0 & - 4\\
 0 & 1 & - 2 & 2\\
 0 & 2 & - 3 & 3
\end{pmatrix}\ \ \ \xrightarrow{ \text{line} 3 \ -=\  \text{line} 2 \times\left( 2 \right)} %
\begin{pmatrix}
 1 & 0 & 0 & - 4\\
 0 & 1 & - 2 & 2\\
 0 & 0 & 1 & -1
\end{pmatrix}\allowdisplaybreaks\\
 &\ \ \ \xrightarrow{ \text{line} 2 \ +=\  \text{line} 3 \times\left( 2 \right)} %
\begin{pmatrix}
 1 & 0 & 0 & - 4\\
 0 & 1 & 0 & 0\\
 0 & 0 & 1 & -1
\end{pmatrix}
\end{align*}
したがって,$(x, y, z)=(-4, 0, -1)$
%%%%%%%%%%%%%%%
\item
\begin{align*}
 \begin{pmatrix}
 -1 & 1 & -1 & 1\\
 8 & - 7 & 6 & - 3\\
 - 5 & 5 & - 6 & 7
\end{pmatrix}\allowdisplaybreaks\\
&\ \ \ \xrightarrow{ \text{line} 1 \ \times=\ \left( - 1 \right)} %
\begin{pmatrix}
 1 & -1 & 1 & -1\\
 8 & - 7 & 6 & - 3\\
 - 5 & 5 & - 6 & 7
\end{pmatrix}\ \ \ \xrightarrow{ \text{line} 2 \ -=\  \text{line} 1 \times\left( 8 \right)} %
\begin{pmatrix}
 1 & -1 & 1 & -1\\
 0 & 1 & - 2 & 5\\
 - 5 & 5 & - 6 & 7
\end{pmatrix}\allowdisplaybreaks\\
&\ \ \ \xrightarrow{ \text{line} 3 \ +=\  \text{line} 1 \times\left( 5 \right)} %
\begin{pmatrix}
 1 & -1 & 1 & -1\\
 0 & 1 & - 2 & 5\\
 0 & 0 & -1 & 2
\end{pmatrix}\ \ \ \xrightarrow{ \text{line} 1 \ +=\  \text{line} 2 } %
\begin{pmatrix}
 1 & 0 & -1 & 4\\
 0 & 1 & - 2 & 5\\
 0 & 0 & -1 & 2
\end{pmatrix}\allowdisplaybreaks\\
&\ \ \ \xrightarrow{ \text{line} 3 \ \times=\ \left( - 1 \right)} %
\begin{pmatrix}
 1 & 0 & -1 & 4\\
 0 & 1 & - 2 & 5\\
 0 & 0 & 1 & - 2
\end{pmatrix}\ \ \ \xrightarrow{ \text{line} 1 \ +=\  \text{line} 3 } %
\begin{pmatrix}
 1 & 0 & 0 & 2\\
 0 & 1 & - 2 & 5\\
 0 & 0 & 1 & - 2
\end{pmatrix}\allowdisplaybreaks\\
&\ \ \ \xrightarrow{ \text{line} 2 \ +=\  \text{line} 3 \times\left( 2 \right)} %
\begin{pmatrix}
 1 & 0 & 0 & 2\\
 0 & 1 & 0 & 1\\
 0 & 0 & 1 & - 2
\end{pmatrix}
\end{align*}
したがって,$(x, y, z)=(2, 1, -2)$
%%%%%%%%%%%%%%%
\item
\begin{align*}
\begin{pmatrix}
 1 & 1 & -1 & 3\\
 - 3 & - 2 & 3 & - 9\\
 - 3 & -1 & 2 & - 8
\end{pmatrix}\\
 &\ \ \ \xrightarrow{ \text{line} 2 \ +=\  \text{line} 1 \times\left( 3 \right)} %
\begin{pmatrix}
 1 & 1 & -1 & 3\\
 0 & 1 & 0 & 0\\
 - 3 & -1 & 2 & - 8
\end{pmatrix}
 \ \ \ \xrightarrow{ \text{line} 3 \ +=\  \text{line} 1 \times\left( 3 \right)} %
\begin{pmatrix}
 1 & 1 & -1 & 3\\
 0 & 1 & 0 & 0\\
 0 & 2 & -1 & 1
\end{pmatrix}\allowdisplaybreaks\\
 &\ \ \ \xrightarrow{ \text{line} 1 \ -=\  \text{line} 2 } %
\begin{pmatrix}
 1 & 0 & -1 & 3\\
 0 & 1 & 0 & 0\\
 0 & 2 & -1 & 1
\end{pmatrix}
 \ \ \ \xrightarrow{ \text{line} 3 \ -=\  \text{line} 2 \times\left( 2 \right)} %
\begin{pmatrix}
 1 & 0 & -1 & 3\\
 0 & 1 & 0 & 0\\
 0 & 0 & -1 & 1
\end{pmatrix}\allowdisplaybreaks\\
 &\ \ \ \xrightarrow{ \text{line} 3 \ \times=\ \left( - 1 \right)} %
\begin{pmatrix}
 1 & 0 & -1 & 3\\
 0 & 1 & 0 & 0\\
 0 & 0 & 1 & -1
\end{pmatrix}
 \ \ \ \xrightarrow{ \text{line} 1 \ +=\  \text{line} 3 } %
\begin{pmatrix}
 1 & 0 & 0 & 2\\
 0 & 1 & 0 & 0\\
 0 & 0 & 1 & -1
\end{pmatrix}
\end{align*}
したがって,$(x, y, z)=(2, 0, -1)$
%%%%%%%%%%%%%%%%%
\setcounter{enumi}{5}
\item
\begin{align*}
\begin{pmatrix}
 1 & - 2 & -1 & - 2\\
 - 2 & 4 & 1 & 2\\
 - 3 & 5 & 3 & 7
\end{pmatrix}\allowdisplaybreaks\\
&\ \ \ \xrightarrow{ \text{line} 2 \ +=\  \text{line} 1 \times\left( 2 \right)} %
\begin{pmatrix}
 1 & - 2 & -1 & - 2\\
 0 & 0 & -1 & - 2\\
 - 3 & 5 & 3 & 7
\end{pmatrix}\ \ \ \xrightarrow{ \text{line} 3 \ +=\  \text{line} 1 \times\left( 3 \right)} %
\begin{pmatrix}
 1 & - 2 & -1 & - 2\\
 0 & 0 & -1 & - 2\\
 0 & -1 & 0 & 1
\end{pmatrix}\allowdisplaybreaks\\
&\ \ \ \xrightarrow{ \text{line} 2 \ \leftrightarrow\  \text{line} 3 } %
\begin{pmatrix}
 1 & - 2 & -1 & - 2\\
 0 & -1 & 0 & 1\\
 0 & 0 & -1 & - 2
\end{pmatrix}\ \ \ \xrightarrow{ \text{line} 2 \ \times=\ \left( - 1 \right)} %
\begin{pmatrix}
 1 & - 2 & -1 & - 2\\
 0 & 1 & 0 & -1\\
 0 & 0 & -1 & - 2
\end{pmatrix}\allowdisplaybreaks\\
&\ \ \ \xrightarrow{ \text{line} 1 \ +=\  \text{line} 2 \times\left( 2 \right)} %
\begin{pmatrix}
 1 & 0 & -1 & - 4\\
 0 & 1 & 0 & -1\\
 0 & 0 & -1 & - 2
\end{pmatrix}\ \ \ \xrightarrow{ \text{line} 3 \ \times=\ \left( - 1 \right)} %
\begin{pmatrix}
 1 & 0 & -1 & - 4\\
 0 & 1 & 0 & -1\\
 0 & 0 & 1 & 2
\end{pmatrix}\allowdisplaybreaks\\
&\ \ \ \xrightarrow{ \text{line} 1 \ +=\  \text{line} 3 } %
\begin{pmatrix}
 1 & 0 & 0 & - 2\\
 0 & 1 & 0 & -1\\
 0 & 0 & 1 & 2
\end{pmatrix}
\end{align*}
したがって,$(x, y, z)=(-2, -1, 2)$
\end{enumerate}
\end{document}
