%%%%%%%%%%%%%%%%%
\element{lasystem}{
\begin{question}{lasystem01}
以下の連立方程式を行列の基本変形によって解きなさい.
\begin{equation*}
\begin{cases}
 x + y - z =3\\
-3x -2y + 3z = -9\\
-3x -y + 2z = -8
\end{cases}
\end{equation*}
%\begin{equation*}
%\begin{pmatrix}
% 1 & 1 & -1 & 3\\
% - 3 & - 2 & 3 & - 9\\
% - 3 & -1 & 2 & - 8
%\end{pmatrix}
%\end{equation*}
\AMCOpen{lines=8, dots=false, framerule=0pt}{\wrongchoice[W]{誤}\scoring{0}\wrongchoice[P]{部}\scoring{1}\correctchoice[C]{正}\scoring{2}}
\end{question}
}

%%%%%%%%%%%%%%%%%
\element{lasystem}{
\begin{question}{lasystem02}
以下の連立方程式を行列の基本変形によって解きなさい.
\begin{equation*}
 \begin{cases}
 -x + y - z = 1\\
8x - 7y + 6z = -3\\
-5x + 5y -6z = 7
\end{cases}
\end{equation*}
%\begin{equation*}
% \begin{pmatrix}
% -1 & 1 & -1 & 1\\
% 8 & - 7 & 6 & - 3\\
% - 5 & 5 & - 6 & 7
%\end{pmatrix}
%\end{equation*}
\AMCOpen{lines=8, dots=false, framerule=0pt}{\wrongchoice[W]{誤}\scoring{0}\wrongchoice[P]{部}\scoring{1}\correctchoice[C]{正}\scoring{2}}
\end{question}
}

%%%%%%%%%%%%%%%%%
\element{lasystem}{
\begin{question}{lasystem03}
以下の連立方程式を行列の基本変形によって解きなさい.
\begin{equation*}
 \begin{cases}
 x - 2y - z = -2\\
 -2x + 4y + z = 2\\
 -3x + 5y + 3z = 7
\end{cases}
\end{equation*}
%\begin{equation*}
%\begin{pmatrix}
% 1 & - 2 & -1 & - 2\\
% - 2 & 4 & 1 & 2\\
% - 3 & 5 & 3 & 7
%\end{pmatrix}
%\end{equation*}
\AMCOpen{lines=8, dots=false, framerule=0pt}{\wrongchoice[W]{誤}\scoring{0}\wrongchoice[P]{部}\scoring{1}\correctchoice[C]{正}\scoring{2}}
\end{question}
}

%%%%%%%%%%%%%%%%%
\element{lasystem}{
\begin{question}{lasystem04}
以下の連立方程式を行列の基本変形によって解きなさい.
\begin{equation*}
 \begin{cases}
x-y+2z=-6\\
-x+3y-5z=9\\
2x+y-z=-7
\end{cases}
\end{equation*}
%\begin{equation*}
%\begin{pmatrix}
% 1 & -1 & 2 & -6\\
% -1 & 3 & -5 & 9\\
% 2 & 1 & -1 & -7
%\end{pmatrix}
%\end{equation*}
\AMCOpen{lines=8, dots=false, framerule=0pt}{\wrongchoice[W]{誤}\scoring{0}\wrongchoice[P]{部}\scoring{1}\correctchoice[C]{正}\scoring{2}}
\end{question}
}

%%%%%%%%%%%%%%%%%
\element{lasystem}{
\begin{question}{lasystem05}
以下の連立方程式を行列の基本変形によって解きなさい.
\begin{equation*}
 \begin{cases}
-7x-y-2z=5\\
-6x-y-2z=5\\
4x+y+3z=-7
\end{cases}
\end{equation*}
%\begin{equation*}
%\begin{pmatrix}
% - 7 & -1 & - 2 & 5\\
% - 6 & -1 & - 2 & 5\\
% 4 & 1 & 3 & - 7
%\end{pmatrix}
%\end{equation*}
\AMCOpen{lines=8, dots=false, framerule=0pt}{\wrongchoice[W]{誤}\scoring{0}\wrongchoice[P]{部}\scoring{1}\correctchoice[C]{正}\scoring{2}}
\end{question}
}

%%%%%%%%%%%%%%%%%
\element{lasystem}{
\begin{question}{lasystem06}
以下の連立方程式を行列の基本変形によって解きなさい.
\begin{equation*}
\begin{cases}
x+6y+3z=5\\
-2x-9y-5z=-8\\
-3x+y-3z=-3
\end{cases}
\end{equation*}
%\begin{equation*}
% \begin{cases}
%  1 & 6 & 3 & 5\\
%  -2 & -9 & -5 & -8\\
%  -3 & 1 & -3 & -3
% \end{cases}
%\end{equation*}
\AMCOpen{lines=8, dots=false, framerule=0pt}{\wrongchoice[W]{誤}\scoring{0}\wrongchoice[P]{部}\scoring{1}\correctchoice[C]{正}\scoring{2}}
\end{question}
}