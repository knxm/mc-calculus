\documentclass[a4paper]{ltjsarticle}
\usepackage{amsmath, amssymb}
\usepackage{tikz, luacode}
\usepackage[box,completemulti,lang=JA]{automultiplechoice}
\newcommand*{\var}[1]{\luaexec{tex.print(#1)}}
\begin{document}
\AMCboxDimensions{shape=oval,width=1.8ex,height=2.5ex}
\AMCcodeVspace=0em
\luaexec{math.randomseed(190605)} % \onecopy{} の外に配置

\begin{luacode*}
function execMaxima(command)
  local maxima_command="echo 'tex1("..command..");' | maxima --very-quiet "
  local handle=io.popen(maxima_command, "r")
  local content=string.gsub(handle:read("*all"), "\n", "")
  handle:close()
  return content
end
\end{luacode*}

%%%%%%%%%%%%%%%%%%%%%%%%%%%%%%%%%%%%%%%%%%%%%%%%%%%%%%%%%%%%%
\element{subst_int}{
\begin{question}{sint01}
\luaexec{
  a=math.random(4, 9);
  f='sin(x)^('..a..')*cos(x)';
  formula=execMaxima(f);
}
不定積分$\displaystyle\int \var{formula}\, dx$を求めなさい.
\AMCOpen{lines=4, dots=false, framerule=0pt}{\wrongchoice[W]{誤}\scoring{0}\wrongchoice[P]{部}\scoring{1}\correctchoice[C]{正}\scoring{2}}
\end{question}}
%%%%%%%%%%%%%%%%%%%%%%%%%%%%%%%%%%%%%%%%%%%%%%%%%%%%%%%%%%%%%
\element{subst_int}{
\begin{question}{sint02}
\luaexec{
  table={1, 3, 5, 7};
  i=math.random(1, 4);
  a=table[i];
  b=math.random(1, 4);
  c=math.random(3, 7);
  g='x^2+'..a..'*x+'..b;
  f='(2*x+'..a..')*(x^2+'..a..'*x+'..b..')^'..c;
  formula=execMaxima(f);
}
不定積分$\displaystyle\int\var{formula}\, dx$を求めなさい.
\AMCOpen{lines=4, dots=false, framerule=0pt}{\wrongchoice[W]{誤}\scoring{0}\wrongchoice[P]{部}\scoring{1}\correctchoice[C]{正}\scoring{2}}
\end{question}
}
%%%%%%%%%%%%%%%%%%%%%%%%%%%%%%%%%%%%%%%%%%%%%%%%%%%%%%%%%%%%%
\element{subst_int}{
\begin{question}{sint03}
\luaexec{
  a=math.random(2, 5);
  b=math.random(1, 9);
  f='exp('..a..'*x)/(exp('..a..'*x)+'..b..')';
  formula=execMaxima(f);
}
不定積分$\displaystyle\int\var{formula}\, dx$を求めなさい.
\AMCOpen{lines=4, dots=false, framerule=0pt}{\wrongchoice[W]{誤}\scoring{0}\wrongchoice[P]{部}\scoring{1}\correctchoice[C]{正}\scoring{2}}
\end{question}
}
%%%%%%%%%%%%%%%%%%%%%%%%%%%%%%%%%%%%%%%%%%%%%%%%%%%%%%%%%%%%%
\element{subst_int}{
\begin{question}{sint04}
\luaexec{
  a=math.random(2, 5);
  b=math.random(1, 9);
  f='log('..a..'*x+'..b..')/('..a..'*x+'..b..')';
  formula=execMaxima(f);
}
不定積分$\displaystyle\int\var{formula}\, dx$を求めなさい.
\AMCOpen{lines=4, dots=false, framerule=0pt}{\wrongchoice[W]{誤}\scoring{0}\wrongchoice[P]{部}\scoring{1}\correctchoice[C]{正}\scoring{2}}
\end{question}
}
%%%%%%%%%%%%%%%%%%%%%%%%%%%%%%%%%%%%%%%%%%%%%%%%%%%%%%%%%%%%%
\element{subst_int}{
\begin{question}{sint05}
\luaexec{
  a=math.random(2, 7);
  f='x/sqrt('..a..'-x)';
  formula=execMaxima(f);
}
不定積分$\displaystyle\int\var{formula}\, dx$を求めなさい.
\AMCOpen{lines=4, dots=false, framerule=0pt}{\wrongchoice[W]{誤}\scoring{0}\wrongchoice[P]{部}\scoring{1}\correctchoice[C]{正}\scoring{2}}
\end{question}
}
%%%%%%%%%%%%%%%%%%%%%%%%%%%%%%%%%%%%%%%%%%%%%%%%%%%%%%%%%%%%%
\element{subst_int}{
\begin{question}{sint06}
\luaexec{
  a=math.random(2, 7);
  b=math.random(2, 7);
  f='x*(x-'..a..')^'..b;
  formula=execMaxima(f);
}
不定積分$\displaystyle\int\var{formula}\, dx$を求めなさい.
\AMCOpen{lines=4, dots=false, framerule=0pt}{\wrongchoice[W]{誤}\scoring{0}\wrongchoice[P]{部}\scoring{1}\correctchoice[C]{正}\scoring{2}}
\end{question}
}

\element{subst_int}{
\begin{question}{sint07}
\luaexec{
  a=math.random(2, 9);
  b=math.random(0, 1);
  c=(-1)^b*a;
  f='x/(x+'..c..')^2';
  formula=execMaxima(f);
}
不定積分$\displaystyle\int\var{formula}\, dx$を求めなさい.
\AMCOpen{lines=4, dots=false, framerule=0pt}{\wrongchoice[W]{誤}\scoring{0}\wrongchoice[P]{部}\scoring{1}\correctchoice[C]{正}\scoring{2}}
\end{question}
}

\onecopy{2}{
%%% ヘッダー開始: %%%   
\noindent{\textbf{応用数学 演習09 \hfill 2019年6月5日}}

\vspace{3ex}

{\small
  \setlength{\parindent}{0pt}\hspace*{\fill}\AMCcodeGridInt{学生番号}{8}\hspace*{\fill}
\begin{minipage}[b]{6.5cm}$\longleftarrow{}$\hspace{0ptplus1cm}
学生番号を左にマークし、下に氏名を記入してください。

\vspace{3ex}

\hfill\namefield{
  \fbox{
    \begin{minipage}{.9\linewidth}
      氏名\vspace*{.5cm}%\dotfill\vspace*{.5cm}\dotfill
      \vspace*{1mm}
    \end{minipage}
  }
}
\hfill\vspace{5ex}
\end{minipage}
\hspace*{\fill}
}
% \multiSymbole{}の記号のある設問の正解は1個とは限りません。 0個の場合や複数の場合があります。

解答欄上部の「誤部正」は採点欄ですので,決して記入しないでください.

\vspace{1ex}
\hrulefill
\vspace{1ex}

%%% ヘッダー終了 %%%
%%%%%%%%%%%%%%%%%%%%%%%%%%%%%%%%%%%%%%%%%%%%%%%%%%%%%%%%%%%%%
部分積分の問題なので,必ず$\varphi(x)$と$\varphi\,'(x)$を記述すること.
解く過程も重視します.

\copygroup{subst_int}{all}
\shufflegroup{all}
\insertgroup[2]{all}

} % End of \onecopy{}
%%%%%%%%%%%%%%%%%%%%%%%%%%%%%%%%%%%%%%%%%%%%%%%%%%%%%%%%%%%%%
\end{document}
