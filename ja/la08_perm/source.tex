\documentclass[a4paper]{ltjsarticle}
\usepackage{amsmath, amssymb}
\usepackage{tikz, luacode}
\usepackage[box,completemulti,lang=JA]{automultiplechoice}
\newcommand*{\var}[1]{\luaexec{tex.print(#1)}}
\begin{document}
\AMCboxDimensions{shape=oval,width=1.8ex,height=2.5ex}
\AMCcodeVspace=0em
\luaexec{math.randomseed(190613)} % \onecopy{} の外に配置

%\begin{luacode*}
%function execMaxima(command)
%  local maxima_command="echo 'tex1("..command..");' | maxima --very-quiet "
%  local handle=io.popen(maxima_command, "r")
%  local content=string.gsub(handle:read("*all"), "\n", "")
%  handle:close()
%  return content
%end
%function execSage(command)
%  local sage_command="sage -c 'print(latex("..command.."))'"
%  local handle=io.popen(sage_command, "r")
%  local content=string.gsub(handle:read("*all"), "\n", "")
%  handle:close()
%  return content
%end
%\end{luacode*}

\onecopy{2}{
%%% ヘッダー開始: %%%   
\noindent{\textbf{線形代数 演習08 \hfill 2019年6月13日}}

\vspace{3ex}

{\small
  \setlength{\parindent}{0pt}\hspace*{\fill}\AMCcodeGridInt{学生番号}{8}\hspace*{\fill}
\begin{minipage}[b]{6.5cm}
$\longleftarrow{}$\hspace{0ptplus1cm}
学生番号を左にマークし、下に氏名とNU-AppsGのメールアドレス(brxxyyzzz)を記入してください。
xxは名前から,yyは入学年度,zzzは学生番号下3桁です.@以降は必要ありません.

\vspace{3ex}

\hfill\namefield{
  \fbox{
    \begin{minipage}{.5\linewidth}
      氏名\vspace*{.5cm}%\dotfill\vspace*{.5cm}\dotfill
      \vspace*{1mm}
    \end{minipage}
  }
  \fbox{
    \begin{minipage}{.5\linewidth}
      NU-AppsG\vspace*{.5cm}%\dotfill\vspace*{.5cm}\dotfill
      \vspace*{1mm}
    \end{minipage}
 }
}
\hfill\vspace{5ex}
\end{minipage}
\hspace*{\fill}
}

\multiSymbole{}の記号のある設問の正解は1個とは限りません。 0個の場合や複数の場合があります。

解答欄上部の「誤部正」は採点欄ですので,決して記入しないでください.

\vspace{1ex}
\hrulefill
\vspace{1ex}

%%% ヘッダー終了 %%%
%%%%%%%%%%%%%%%%%%%%%%%%%%%%%%%%%%%%%%%%%%%%%%%%%%%%%%%%%%%%%%
%\begin{question}{perm01}
%置換\(\sigma = \begin{pmatrix}
%		1 & 2 & 3\\
%		3 & 1 & 2		
%	       \end{pmatrix}\)
%を集合の関係として表記しなさい.
%\AMCOpen{lines=1, dots=false, framerule=0pt}{\wrongchoice[W]{誤}\scoring{0}\wrongchoice[P]{部}\scoring{1}\correctchoice[C]{正}\scoring{2}}
%\end{question}
%%%%%%%%%%%%%%%%%%%%%%%%%%%%%%%%%%%%%%%%%%%%%%%%%%%%%%%%%%%%%
\begin{question}{perm02}
\(\sigma = \begin{pmatrix}
	    1 & 2 & 3 & 4\\
	    4 & 1 & 3 & 2
	   \end{pmatrix}\)
を巡回置換として表記しなさい.
\AMCOpen{lines=1, dots=false, framerule=0pt}{\wrongchoice[W]{誤}\scoring{0}\wrongchoice[P]{部}\scoring{1}\correctchoice[C]{正}\scoring{2}}
\end{question}
%%%%%%%%%%%%%%%%%%%%%%%%%%%%%%%%%%%%%%%%%%%%%%%%%%%%%%%%%%%%%
\begin{questionmult}{perm03}
つぎの置換のうち,互換はどれか.

\begin{choiceshoriz}
\correctchoice{\(\begin{pmatrix}2 & 1 & 3\\1 & 2 & 3\end{pmatrix}\)}
\correctchoice{\(\begin{pmatrix}1 & 2 & 3\\1 & 3 & 2\end{pmatrix}\)}
\wrongchoice{\(\begin{pmatrix}1 & 2 & 3\\3 & 1 & 2\end{pmatrix}\)}
\wrongchoice{\(\begin{pmatrix}2 & 3 & 1\\1 & 2 & 3\end{pmatrix}\)}
\wrongchoice{\(\begin{pmatrix}1 & 2 & 3\\1 & 2 & 3\end{pmatrix}\)}
\end{choiceshoriz}
\end{questionmult}
%%%%%%%%%%%%%%%%%%%%%%%%%%%%%%%%%%%%%%%%%%%%%%%%%%%%%%%%%%%%%%
\begin{question}{perm04}
集合\(X = \{a, b, c\}\)に対する3次の置換を全て書き出しなさい.
\AMCOpen{lines=2, dots=false, framerule=0pt}{\wrongchoice[W]{誤}\scoring{0}\wrongchoice[P]{部}\scoring{1}\correctchoice[C]{正}\scoring{2}}
\end{question}
%%%%%%%%%%%%%%%%%%%%%%%%%%%%%%%%%%%%%%%%%%%%%%%%%%%%%%%%%%%%%%
} % End of \onecopy{}
\end{document}
