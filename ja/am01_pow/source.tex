\documentclass[a4paper]{ltjsarticle}
\usepackage{tikz}
\usepackage{luacode}
\usepackage[box,completemulti,lang=JA]{automultiplechoice}
\newcommand*{\var}[1]{\luadirect{tex.print(#1)}}
%%% document開始:
\begin{document}
\AMCboxDimensions{shape=oval,width=1.8ex,height=2.5ex}
% 乱数の種を設定する必要あり
\luaexec{math.randomseed(190410)}

\onecopy{2}{

%%% 試験問題用紙ヘッダー開始:    

\noindent{\bf 応用数学 演習01 \hfill 2019年4月10日}

\vspace{3ex}

{\small
  \setlength{\parindent}{0pt}\hspace*{\fill}\AMCcodeGridInt{学生番号}{8}\hspace*{\fill}
\begin{minipage}[b]{6.5cm}$\longleftarrow{}$\hspace{0ptplus1cm}
学生番号を左にマークし、下に氏名を記入してください。

\vspace{3ex}

\hfill\namefield{
  \fbox{
    \begin{minipage}{.9\linewidth}
      氏名:\vspace*{.5cm}\dotfill\vspace*{.5cm}\dotfill
      \vspace*{1mm}
    \end{minipage}
  }
}
\hfill\vspace{5ex}
\end{minipage}
\hspace*{\fill}
}

\multiSymbole{}の記号のある設問の正解は1個とは限りません。 0個の場合や複数の場合
があります。

\vspace{1ex}
\hrulefill
\vspace{1ex}

%%% ヘッダー終了
%%%%%%%%%%%%%%%%%%%%%%%%%%%%%%%%%%%%%%%%%%%%%%%%%%%%%%%%%%%%%
\begin{question}{power01}
\luaexec{
  a=math.random(5, 9);
  if (a > 7) then b=3 else b=4 end;
  correct=a^b;
  wrong1=a^(b+1);
  wrong2=a^(b-1);
  wrong3=a^b+1;
  wrong4=a^b-1;
}
  \(\var{a}^\var{b}\)と等しいものを選べ.

  \begin{choiceshoriz}
    \correctchoice{\(\var{correct}\)}
    \wrongchoice{\(\var{wrong1}\)}
    \wrongchoice{\(\var{wrong2}\)}
    \wrongchoice{\(\var{wrong3}\)}
    \wrongchoice{\(\var{wrong4}\)}
  \end{choiceshoriz}
\end{question}
%%%%%%%%%%%%%%%%%%%%%%%%%%%%%%%%%%%%%%%%%%%%%%%%%%%%%%%%%%%%%
\begin{question}{power02}
\luaexec{
  a=math.random(2, 9);
}
  \(\var{a}^{-1}\)と等しいものを選べ.

  \begin{choiceshoriz}
    \correctchoice{\(\frac{1}{\var{a}}\)}
    \wrongchoice{\(-\var{a}\)}
    \wrongchoice{\(\frac{1}{\var{a+1}}\)}
    \wrongchoice{\(-\var{a+1}\)}
    \wrongchoice{\(\sqrt{\var{a}}\)}
  \end{choiceshoriz}
\end{question}
%%%%%%%%%%%%%%%%%%%%%%%%%%%%%%%%%%%%%%%%%%%%%%%%%%%%%%%%%%%%%
\begin{question}{power03}
\luaexec{
  a=math.random(2, 9);
}
  \(\var{a}^{\frac{1}{2}}\)と等しいものを選べ.

  \begin{choiceshoriz}
    \correctchoice{\(\sqrt{\var{a}}\)}
    \wrongchoice{\(-\var{a}\)}
    \wrongchoice{\(\frac{1}{\var{a+1}}\)}
    \wrongchoice{\(-\var{a+1}\)}
    \wrongchoice{\(\frac{1}{\var{a}}\)}
  \end{choiceshoriz}
\end{question}
%%%%%%%%%%%%%%%%%%%%%%%%%%%%%%%%%%%%%%%%%%%%%%%%%%%%%%%%%%%%%
\begin{questionmult}{power04}
  \luaexec{
    a=math.random(2, 9);
    b=math.random(2, 5);
  }
  \(\left(\frac{1}{\var{a}}\right)^{-\var{b}}\)と等しいものを全て選べ.
  \begin{choiceshoriz}
    \correctchoice{\(\var{a}^\var{b}\)}
    \correctchoice{\(\left(\frac{1}{\var{a}^\var{b}}\right)^{-1}\)}
    \wrongchoice{\(\left(\var{a}^\var{b}\right)^{-1}\)}
    \wrongchoice{\(\frac{1}{\var{a}^\var{b}}\)}
    \wrongchoice{\(\sqrt[\var{b}]{\var{a}}\)}
  \end{choiceshoriz}   
\end{questionmult}

%%%%%%%%%%%%%%%%%%%%%%%%%%%%%%%%%%%%%%%%%%%%%%%%%%%%%%%%%%%%%
\begin{question}{power05}
  \luaexec{
    a=math.random(5, 9);
    b=math.random(5, 9);
    correct=a+b;
    wrong1=math.abs(a-b)+1;
    wrong2=a*b;
    wrong3=a+b+1;
    wrong4=a*b-1;
  }
  \(2^{\var{a}}\times{}2^{\var{b}}\)と等しいものを選べ.
  \begin{choiceshoriz}
    \correctchoice{\(2^{\var{correct}}\)}
    \wrongchoice{\(2^{\var{wrong1}}\)}
    \wrongchoice{\(2^{\var{wrong2}}\)}
    \wrongchoice{\(2^{\var{wrong3}}\)}
    \wrongchoice{\(2^{\var{wrong4}}\)}
  \end{choiceshoriz}   
\end{question}
%%%%%%%%%%%%%%%%%%%%%%%%%%%%%%%%%%%%%%%%%%%%%%%%%%%%%%%%%%%%%
\begin{question}{log01}
  \luaexec{
    a=math.random(4, 7);
    b=math.random(2, 3);
    correct=a;
    wrong1=b;
    wrong2=a+b;
    wrong3=a*b;
    wrong4=b+1;
  }
  方程式\(\log_{x}\var{a^b}=\var{b}\)の解を求めよ.
  \begin{choiceshoriz}
    \correctchoice{\(\var{correct}\)}
    \wrongchoice{\(\var{wrong1}\)}
    \wrongchoice{\(\var{wrong2}\)}
    \wrongchoice{\(\var{wrong3}\)}
    \wrongchoice{\(\var{wrong4}\)}
  \end{choiceshoriz}   
\end{question}
%%%%%%%%%%%%%%%%%%%%%%%%%%%%%%%%%%%%%%%%%%%%%%%%%%%%%%%%%%%%%
\begin{question}{log02}
  \luaexec{
    a=math.random(4, 8);
    b=math.random(-3, 3);
    wrong1=a^(b-1);
    wrong2=a^(b+1);
    wrong3=a*b;
    wrong4=a;
  }
  方程式\(\log_{\var{a}}x=\var{b}\)の解を求めよ.
  \begin{choiceshoriz}
    \correctchoice{\(\var{a}^{\var{b}}\)}
    \wrongchoice{\(\var{a-1}^{\var{b+1}}\)}
    \wrongchoice{\(\var{a}^{\var{b-1}}\)}
    \wrongchoice{\(\var{a}^{\var{b+1}}\)}
    \wrongchoice{\(\var{a-1}^{\var{b}}\)}
  \end{choiceshoriz}   
\end{question}
%%%%%%%%%%%%%%%%%%%%%%%%%%%%%%%%%%%%%%%%%%%%%%%%%%%%%%%%%%%%%
\begin{question}{log03}
  \luaexec{
    a=math.random(2, 5);
    b=math.random(2, 10);
    c=math.random(2, 4);
    d=math.random(1, 8);
    e=a^c;
    table={-3, -2, -1, 1, 2, 3};
    r=math.random(1, 3);
    correct=(b-d)/2;
    wrong1=correct+table[r];
    wrong2=correct+table[r+1];
    wrong3=correct+table[r+2];
    wrong4=correct+table[r+3];
  }
  方程式\(\log_{\var{a}}(\var{b}-x)=\var{c}\log_{\var{e}}(x+\var{d})\)の解を求めよ.
  \begin{choiceshoriz}
    \correctchoice{\(\var{correct}\)}
    \wrongchoice{\(\var{wrong1}\)}
    \wrongchoice{\(\var{wrong2}\)}
    \wrongchoice{\(\var{wrong3}\)}
    \wrongchoice{\(\var{wrong4}\)}
  \end{choiceshoriz}   
\end{question}
%%%%%%%%%%%%%%%%%%%%%%%%%%%%%%%%%%%%%%%%%%%%%%%%%%%%%%%%%%%%%

% \AMCaddpagesto{3} 
}

\end{document}


\end{document}

