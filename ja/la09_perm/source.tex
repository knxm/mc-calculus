\documentclass[a4paper]{ltjsarticle}
\usepackage{amsmath, amssymb}
\usepackage{tikz, luacode}
\usepackage[box,completemulti,lang=JA,noshuffle]{automultiplechoice}
\newcommand*{\var}[1]{\luaexec{tex.print(#1)}}
\DeclareMathOperator{\sgn}{sgn}
\begin{document}
\AMCboxDimensions{shape=oval,width=1.8ex,height=2.5ex}
\AMCcodeVspace=0em
%\luaexec{math.randomseed(19620)} % \onecopy{} の外に配置

%\begin{luacode*}
%function execMaxima(command)
%  local maxima_command="echo 'tex1("..command..");' | maxima --very-quiet "
%  local handle=io.popen(maxima_command, "r")
%  local content=string.gsub(handle:read("*all"), "\n", "")
%  handle:close()
%  return content
%end
%function execSage(command)
%  local sage_command="sage -c 'print(latex("..command.."))'"
%  local handle=io.popen(sage_command, "r")
%  local content=string.gsub(handle:read("*all"), "\n", "")
%  handle:close()
%  return content
%end
%\end{luacode*}
%%%%%%%%%%%%%%%%%%%%%%%%%%%%%%%%%%%%%%%%%%%%%%%%%%%%%%%%%%%%%
\element{perm}{
\begin{question}{perm05-1}
置換\(\sigma = \begin{pmatrix}
	  1 & 2 & 3 & 4 & 5\\
	  5 & 1 & 4 & 3 & 2

	       \end{pmatrix}\)
を巡回置換の積で表しなさい.
\AMCOpen{lines=2, dots=false, framerule=0pt}{\wrongchoice[W]{誤}\scoring{0}\wrongchoice[P]{部}\scoring{1}\correctchoice[C]{正}\scoring{2}}
\end{question}
}
%%%%%%%%%%%%%%%%%%%%%%%%%%%%%%%%%%%%%%%%%%%%%%%%%%%%%%%%%%%%%
\element{perm}{
\begin{question}{perm05-2}
置換\(\sigma = \begin{pmatrix}
	  1 & 2 & 3 & 4 & 5\\
	  4 & 1 & 5 & 2 & 3

	       \end{pmatrix}\)
を巡回置換の積で表しなさい.
\AMCOpen{lines=2, dots=false, framerule=0pt}{\wrongchoice[W]{誤}\scoring{0}\wrongchoice[P]{部}\scoring{1}\correctchoice[C]{正}\scoring{2}}
\end{question}
}
%%%%%%%%%%%%%%%%%%%%%%%%%%%%%%%%%%%%%%%%%%%%%%%%%%%%%%%%%%%%%
\element{perm}{
\begin{question}{perm05-3}
置換\(\sigma = \begin{pmatrix}
	  1 & 2 & 3 & 4 & 5\\
	  4 & 5 & 2 & 1 & 3

	       \end{pmatrix}\)
を巡回置換の積で表しなさい.
\AMCOpen{lines=2, dots=false, framerule=0pt}{\wrongchoice[W]{誤}\scoring{0}\wrongchoice[P]{部}\scoring{1}\correctchoice[C]{正}\scoring{2}}
\end{question}
}
%%%%%%%%%%%%%%%%%%%%%%%%%%%%%%%%%%%%%%%%%%%%%%%%%%%%%%%%%%%%%
\element{perm}{
\begin{question}{perm05-4}
置換\(\sigma = \begin{pmatrix}
	  1 & 2 & 3 & 4 & 5\\
	  5 & 3 & 2 & 1 & 4

	       \end{pmatrix}\)
を巡回置換の積で表しなさい.
\AMCOpen{lines=2, dots=false, framerule=0pt}{\wrongchoice[W]{誤}\scoring{0}\wrongchoice[P]{部}\scoring{1}\correctchoice[C]{正}\scoring{2}}
\end{question}
}


\onecopy{2}{
%%% ヘッダー開始: %%%   
\noindent{\textbf{線形代数 演習09 \hfill 2019年6月20日}}

\vspace{3ex}

{\small
  \setlength{\parindent}{0pt}\hspace*{\fill}\AMCcodeGridInt{学生番号}{8}\hspace*{\fill}
\begin{minipage}[b]{6.5cm}
$\longleftarrow{}$\hspace{0ptplus1cm}
学生番号を左にマークし、下に氏名とNU-AppsGのメールアドレス(brxxyyzzz)を記入してください。
xxは名前から,yyは入学年度,zzzは学生番号下3桁です.@以降は必要ありません.

\vspace{3ex}

\hfill\namefield{
  \fbox{
    \begin{minipage}{.5\linewidth}
      氏名\vspace*{.5cm}%\dotfill\vspace*{.5cm}\dotfill
      \vspace*{1mm}
    \end{minipage}
  }
  \fbox{
    \begin{minipage}{.5\linewidth}
      NU-AppsG\vspace*{.5cm}%\dotfill\vspace*{.5cm}\dotfill
      \vspace*{1mm}
    \end{minipage}
 }
}
\hfill\vspace{5ex}
\end{minipage}
\hspace*{\fill}
}
% \multiSymbole{}の記号のある設問の正解は1個とは限りません。 0個の場合や複数の場合があります。

解答欄上部の「誤部正」は採点欄ですので,決して記入しないでください.

\vspace{1ex}
\hrulefill
\vspace{1ex}

%%% ヘッダー終了 %%%
%%%%%%%%%%%%%%%%%%%%%%%%%%%%%%%%%%%%%%%%%%%%%%%%%%%%%%%%%%%%%
\shufflegroup{perm}
\insertgroup[1]{perm}

\begin{question}{perm06}
得られた巡回置換を互換の積として表しなさい.
\AMCOpen{lines=2, dots=false, framerule=0pt}{\wrongchoice[W]{誤}\scoring{0}\wrongchoice[P]{部}\scoring{1}\correctchoice[C]{正}\scoring{2}}
\end{question}
%%%%%%%%%%%%%%%%%%%%%%%%%%%%%%%%%%%%%%%%%%%%%%%%%%%%%%%%%%%%%
\begin{question}{perm07}
置換$\sigma$を互換の積として表しなさい.
\AMCOpen{lines=2, dots=false, framerule=0pt}{\wrongchoice[W]{誤}\scoring{0}\wrongchoice[P]{部}\scoring{1}\correctchoice[C]{正}\scoring{2}}
\end{question}
%%%%%%%%%%%%%%%%%%%%%%%%%%%%%%%%%%%%%%%%%%%%%%%%%%%%%%%%%%%%%
\begin{question}{perm08}
\(\sgn\sigma\)を求めなさい.
\begin{choiceshoriz}
\wrongchoice{$-3$}
\wrongchoice{$-2$}
\correctchoice{$-1$}
\wrongchoice{$0$}
\wrongchoice{$1$}
\wrongchoice{$2$}
\wrongchoice{$3$}
\end{choiceshoriz}
\end{question}
}
\end{document}