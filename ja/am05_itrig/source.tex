\documentclass[a4paper]{ltjsarticle}
\usepackage{amsmath, amssymb}
\usepackage{tikz}
\usepackage{luacode}
\usepackage[box,completemulti,lang=JA]{automultiplechoice}
\newcommand*{\var}[1]{\luaexec{tex.print(#1)}}
\begin{document}
\AMCboxDimensions{shape=oval,width=1.8ex,height=2.5ex}
\luadirect{math.randomseed(190508)} % \onecopy{} の外に配置

\begin{luacode*}
function execSage(command)
  local sage_command="sage -c 'print(latex("..command.."))'"
  local handle=io.popen(sage_command, "r")
  local content=string.gsub(handle:read("*all"), "\n", "")
  handle:close()
  return content
end
function execMaxima(command)
  local maxima_command="echo 'tex1("..command..");' | maxima --very-quiet "
  local handle=io.popen(maxima_command, "r")
  local content=string.gsub(handle:read("*all"), "\n", "")
  handle:close()
  return content
end
\end{luacode*}


\onecopy{2}{

%%% ヘッダー開始: %%%   
\noindent{\bf 応用数学 演習05 \hfill 2019年5月8日}

\vspace{3ex}

{\small
  \setlength{\parindent}{0pt}\hspace*{\fill}\AMCcodeGridInt{学生番号}{8}\hspace*{\fill}
\begin{minipage}[b]{6.5cm}$\longleftarrow{}$\hspace{0ptplus1cm}
学生番号を左にマークし、下に氏名を記入してください。

\vspace{3ex}

\hfill\namefield{
  \fbox{
    \begin{minipage}{.9\linewidth}
      氏名\vspace*{.5cm}%\dotfill\vspace*{.5cm}\dotfill
      \vspace*{1mm}
    \end{minipage}
  }
}
\hfill\vspace{5ex}
\end{minipage}
\hspace*{\fill}
}

\multiSymbole{}の記号のある設問の正解は1個とは限りません。 0個の場合や複数の場合があります。

\vspace{1ex}
\hrulefill
\vspace{1ex}

%%% ヘッダー終了 %%%
%%%%%%%%%%%%%%%%%%%%%%%%%%%%%%%%%%%%%%%%%%%%%%%%%%%%%%%%%%%%%
\begin{questionmult}{itrig01}
\luaexec{
  table={'-2*\%pi', '-11*\%pi/6', '-7*\%pi/4', '-5*\%pi/3', '-3*\%pi/2', '-4*\%pi/3', '-5*\%pi/4', '-7*\%pi/6', '-\%pi', '-5*\%pi/6', '-3*\%pi/4', '-2*\%pi/3', '-\%pi/2', '-\%pi/3', '-\%pi/4', '-\%pi/6', '0', '\%pi/6', '\%pi/4', '\%pi/3', '\%pi/2', '2*\%pi/3', '3*\%pi/4', '5*\%pi/6', '\%pi', '7*\%pi/6', '5*\%pi/4', '4*\%pi/3', '3*\%pi/2', '5*\%pi/3', '7*\%pi/4', '11*\%pi/6', '2*\%pi'}
  i=math.random(13, 21);
  angle=table[i];
  y=execMaxima('sin('..angle..')');
  correct=execMaxima(angle);
  wrong={}
  if i>16 then for k=0, 7 do wrong[k+1]=execMaxima(table[math.fmod(i+k, 33)+1]) end;
  else for k=0, 7 do wrong[k+1]=execMaxima(table[math.fmod(i-k, 33)-1]) end; end;
}
$\arcsin(x)$は逆正弦函数とする.
$\arcsin\left(\var{y}\right)$の主値を求めなさい.
  \begin{choiceshoriz}
    \correctchoice{\(\var{correct}\)}
    \wrongchoice{\(\var{wrong[1]}\)}
    \wrongchoice{\(\var{wrong[2]}\)}
    \wrongchoice{\(\var{wrong[3]}\)}
    \wrongchoice{\(\var{wrong[4]}\)}
    \wrongchoice{\(\var{wrong[5]}\)}
    \wrongchoice{\(\var{wrong[6]}\)}
    \wrongchoice{\(\var{wrong[7]}\)}
    \wrongchoice{\(\var{wrong[8]}\)}
  \end{choiceshoriz}
\end{questionmult}
%%%%%%%%%%%%%%%%%%%%%%%%%%%%%%%%%%%%%%%%%%%%%%%%%%%%%%%%%%%%%
\begin{questionmult}{itrig02}
\luaexec{
  table={'-2*\%pi', '-11*\%pi/6', '-7*\%pi/4', '-5*\%pi/3', '-3*\%pi/2', '-4*\%pi/3', '-5*\%pi/4', '-7*\%pi/6', '-\%pi', '-5*\%pi/6', '-3*\%pi/4', '-2*\%pi/3', '-\%pi/2', '-\%pi/3', '-\%pi/4', '-\%pi/6', '0', '\%pi/6', '\%pi/4', '\%pi/3', '\%pi/2', '2*\%pi/3', '3*\%pi/4', '5*\%pi/6', '\%pi', '7*\%pi/6', '5*\%pi/4', '4*\%pi/3', '3*\%pi/2', '5*\%pi/3', '7*\%pi/4', '11*\%pi/6', '2*\%pi'}
  i=math.random(17, 25); % 0 <= x <= pi
  angle=table[i];
  y=execMaxima('cos('..angle..')');
  correct=execMaxima(angle);
  wrong={}
  if i>20 then for k=0, 7 do wrong[k+1]=execMaxima(table[math.fmod(i+k, 33)+1]) end;
  else for k=0, 7 do wrong[k+1]=execMaxima(table[math.fmod(i-k, 33)-1]) end; end;
}
$\arccos(x)$は逆余弦函数とする.
$\arccos\left(\var{y}\right)$の主値を求めなさい.
  \begin{choiceshoriz}
    \correctchoice{\(\var{correct}\)}
    \wrongchoice{\(\var{wrong[1]}\)}
    \wrongchoice{\(\var{wrong[2]}\)}
    \wrongchoice{\(\var{wrong[3]}\)}
    \wrongchoice{\(\var{wrong[4]}\)}
    \wrongchoice{\(\var{wrong[5]}\)}
    \wrongchoice{\(\var{wrong[6]}\)}
    \wrongchoice{\(\var{wrong[7]}\)}
    \wrongchoice{\(\var{wrong[8]}\)}
  \end{choiceshoriz}
\end{questionmult}
%%%%%%%%%%%%%%%%%%%%%%%%%%%%%%%%%%%%%%%%%%%%%%%%%%%%%%%%%%%%%
\begin{questionmult}{itrig03}
\luaexec{
  table={'-2*\%pi', '-11*\%pi/6', '-7*\%pi/4', '-5*\%pi/3', '-3*\%pi/2', '-4*\%pi/3', '-5*\%pi/4', '-7*\%pi/6', '-\%pi', '-5*\%pi/6', '-3*\%pi/4', '-2*\%pi/3', '-\%pi/2', '-\%pi/3', '-\%pi/4', '-\%pi/6', '0', '\%pi/6', '\%pi/4', '\%pi/3', '\%pi/2', '2*\%pi/3', '3*\%pi/4', '5*\%pi/6', '\%pi', '7*\%pi/6', '5*\%pi/4', '4*\%pi/3', '3*\%pi/2', '5*\%pi/3', '7*\%pi/4', '11*\%pi/6', '2*\%pi'}
  i=math.random(14, 20);
  angle=table[i];
  y=execMaxima('tan('..angle..')');
  correct=execMaxima(angle);
  wrong={}
  if i>16 then for k=0, 7 do wrong[k+1]=execMaxima(table[math.fmod(i+k, 33)+1]) end;
  else for k=0, 7 do wrong[k+1]=execMaxima(table[math.fmod(i-k, 33)-1]) end; end;
}
$\arctan(x)$は逆正接函数とする.
$\arctan\left(\var{y}\right)$の主値を求めなさい.
  \begin{choiceshoriz}
    \correctchoice{\(\var{correct}\)}
    \wrongchoice{\(\var{wrong[1]}\)}
    \wrongchoice{\(\var{wrong[2]}\)}
    \wrongchoice{\(\var{wrong[3]}\)}
    \wrongchoice{\(\var{wrong[4]}\)}
    \wrongchoice{\(\var{wrong[5]}\)}
    \wrongchoice{\(\var{wrong[6]}\)}
    \wrongchoice{\(\var{wrong[7]}\)}
    \wrongchoice{\(\var{wrong[8]}\)}
  \end{choiceshoriz}
\end{questionmult}

%%%%%%%%%%%%%%%%%%%%%%%%%%%%%%%%%%%%%%%%%%%%%%%%%%%%%%%%%%%%%
\begin{questionmult}{diff16}
\luaexec{
  a=math.random(2, 7);
  g='x/sqrt('..a..')';
  f='asin('..g..')';
  h='acos('..g..')';
  formula=execMaxima(f);
  correct1=execMaxima('ratsimp(diff('..f..', x))');
  correct2=execMaxima('diff('..f..', x)');
  wrong1=execMaxima(h);
  wrong2=execMaxima(h..'/sqrt('..a..')');
  wrong3=execMaxima('diff('..f..', x)*sqrt('..a..')');
}
函数 \(f(x)=\var{formula}\)の導函数\(f'(x)\)を求めなさい.
  \begin{choiceshoriz}
   \correctchoice{\(\var{correct1}\)}
   \correctchoice{\(\var{correct2}\)}
   \wrongchoice{\(\var{wrong1}\)}
   \wrongchoice{\(\var{wrong2}\)}
   \wrongchoice{\(\var{wrong3}\)}
  \end{choiceshoriz}

\end{questionmult}

%%%%%%%%%%%%%%%%%%%%%%%%%%%%%%%%%%%%%%%%%%%%%%%%%%%%%%%%%%%%%
\begin{questionmult}{diff17}
\luaexec{
  a=math.random(2, 7);
  g='x/sqrt('..a..')';
  f='acos('..g..')';
  h='-asin('..g..')';
  formula=execMaxima(f);
  correct1=execMaxima('ratsimp(diff('..f..', x))');
  correct2=execMaxima('diff('..f..', x)');
  wrong1=execMaxima(h);
  wrong2=execMaxima(h..'/sqrt('..a..')');
  wrong3=execMaxima('diff('..f..', x)*sqrt('..a..')');
}
函数 \(f(x)=\var{formula}\)の導函数\(f'(x)\)を求めなさい.
  \begin{choiceshoriz}
   \correctchoice{\(\var{correct1}\)}
   \correctchoice{\(\var{correct2}\)}
   \wrongchoice{\(\var{wrong1}\)}
   \wrongchoice{\(\var{wrong2}\)}
   \wrongchoice{\(\var{wrong3}\)}
  \end{choiceshoriz}

\end{questionmult}

%%%%%%%%%%%%%%%%%%%%%%%%%%%%%%%%%%%%%%%%%%%%%%%%%%%%%%%%%%%%%
\begin{questionmult}{diff18}
\luaexec{
  a=math.random(2, 7);
  g='x/'..a;
  f='atan('..g..')';
  h='1/cos('..g..')^2';
  formula=execMaxima(f);
  correct1=execMaxima('ratsimp(diff('..f..', x))');
  correct2=execMaxima('diff('..f..', x)');
  wrong1=execMaxima(h);
  wrong2=execMaxima(h..'/'..a);
  wrong3=execMaxima('diff('..f..', x)*'..a);
}
函数 \(f(x)=\var{formula}\)の導函数\(f'(x)\)を求めなさい.
  \begin{choiceshoriz}
   \correctchoice{\(\var{correct1}\)}
   \correctchoice{\(\var{correct2}\)}
   \wrongchoice{\(\var{wrong1}\)}
   \wrongchoice{\(\var{wrong2}\)}
   \wrongchoice{\(\var{wrong3}\)}
  \end{choiceshoriz}

\end{questionmult}


%%%%%%%%%%%%%%%%%%%%%%%%%%%%%%%%%%%%%%%%%%%%%%%%%%%%%%%%%%%%%

} % end of \onecopy

\end{document}

  wangle={}
  for k=1,8 do j=i+k; wangle[k]=yarray[(i+k)%9+1]; end;
  wrong1=execMaxima('asin('..wangle[1]..')');
    \wrongchoice{\(\var{wrong[1]}\)}

  wangle={};
  for k=0,7 do; wrong[k]=execMaxima('asin('..yarray[(i+k)%9+1]..')'); end;


  wrong1=execMaxima(angle_array[math.fmod(i+1, 9)+1]);
  wrong2=execMaxima(angle_array[math.fmod(i+2, 9)+1]);
  wrong3=execMaxima(angle_array[math.fmod(i+3, 9)+1]);
  wrong4=execMaxima(angle_array[math.fmod(i+4, 9)+1]);
  wrong5=execMaxima(angle_array[math.fmod(i+5, 9)+1]);
  wrong6=execMaxima(angle_array[math.fmod(i+6, 9)+1]);
  wrong7=execMaxima(angle_array[math.fmod(i+7, 9)+1]);
  wrong8=execMaxima(angle_array[math.fmod(i+8, 9)+1]);


\begin{question}{itrig01}
\luaexec{
  yarray={'-1', '-sqrt(3)/2', '-1/sqrt(2)', '-1/2', 0, '1/2', '1/sqrt(2)', 'sqrt(3)/2', '1'};
  i=math.random(1, 9);
  angle=yarray[i];
  y=execMaxima(angle);
  correct=execMaxima('asin('..angle..')');
  wrong1=execMaxima('asin('..yarray[math.fmod(i+1, 9)+1]..')');
  wrong2=execMaxima('asin('..yarray[math.fmod(i+2, 9)+1]..')');
  wrong3=execMaxima('asin('..yarray[math.fmod(i+3, 9)+1]..')');
  wrong4=execMaxima('asin('..yarray[math.fmod(i+4, 9)+1]..')');
  wrong5=execMaxima('asin('..yarray[math.fmod(i+5, 9)+1]..')');
  wrong6=execMaxima('asin('..yarray[math.fmod(i+6, 9)+1]..')');
  wrong7=execMaxima('asin('..yarray[math.fmod(i+7, 9)+1]..')');
  wrong8=execMaxima('asin('..yarray[math.fmod(i+8, 9)+1]..')');
}
$\arcsin(x)$は逆正弦函数とする.
方程式 $\arcsin(x)=\var{y} \,(-1\leqq x\leqq 1)$の解を全て選択しなさい.
  \begin{choiceshoriz}
    \correctchoice{\(\var{correct}\)}
    \wrongchoice{\(\var{wrong1}\)}
    \wrongchoice{\(\var{wrong2}\)}
    \wrongchoice{\(\var{wrong3}\)}
    \wrongchoice{\(\var{wrong4}\)}
    \wrongchoice{\(\var{wrong5}\)}
    \wrongchoice{\(\var{wrong6}\)}
    \wrongchoice{\(\var{wrong7}\)}
    \wrongchoice{\(\var{wrong8}\)}
  \end{choiceshoriz}
\end{question}
