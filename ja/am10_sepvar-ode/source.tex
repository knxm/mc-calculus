\documentclass[a4paper]{ltjsarticle}
\usepackage{amsmath, amssymb}
\usepackage{tikz, luacode}
\usepackage[box,completemulti,lang=JA]{automultiplechoice}
\newcommand*{\var}[1]{\luaexec{tex.print(#1)}}
\begin{document}
\AMCboxDimensions{shape=oval,width=1.8ex,height=2.5ex}
\AMCcodeVspace=0em
\luaexec{math.randomseed(190612)} % \onecopy{} の外に配置

\begin{luacode*}
function execMaxima(command)
  local maxima_command="echo 'tex1("..command..");' | maxima --very-quiet "
  local handle=io.popen(maxima_command, "r")
  local content=string.gsub(handle:read("*all"), "\n", "")
  handle:close()
  return content
end
\end{luacode*}

%%%%%%%%%%%%%%%%%%%%%%%%%%%%%%%%%%%%%%%%%%%%%%%%%%%%%%%%%%%%%
\element{sepvar_ode}{
\begin{question}{svode02}
\luaexec{
  a=math.random(2, 9);
  f='-(t+'..a..')*x^2';
  formula=execMaxima(f);
}
微分方程式 \(\displaystyle\frac{dx}{dt} = \var{formula}\) の一般解を求めなさい.

\AMCOpen{lines=4, dots=false, framerule=0pt}{\wrongchoice[W]{誤}\scoring{0}\wrongchoice[P]{部}\scoring{1}\correctchoice[C]{正}\scoring{2}}
\end{question}
}
%%%%%%%%%%%%%%%%%%%%%%%%%%%%%%%%%%%%%%%%%%%%%%%%%%%%%%%%%%%%%
\element{sepvar_ode}{
\begin{question}{svode01}
\luaexec{
  a=math.random(2, 9);
  f='t^'..a..'*x/('..a..'+1)';
  formula=execMaxima(f);
}
微分方程式 \(\displaystyle\frac{dx}{dt} = \var{formula}\) の一般解を求めなさい.

\AMCOpen{lines=4, dots=false, framerule=0pt}{\wrongchoice[W]{誤}\scoring{0}\wrongchoice[P]{部}\scoring{1}\correctchoice[C]{正}\scoring{2}}
\end{question}
}

\onecopy{2}{
%%% ヘッダー開始: %%%   
\noindent{\textbf{応用数学 演習10 \hfill 2019年6月12日}}

\vspace{3ex}

{\small
  \setlength{\parindent}{0pt}\hspace*{\fill}\AMCcodeGridInt{学生番号}{8}\hspace*{\fill}
\begin{minipage}[b]{6.5cm}$\longleftarrow{}$\hspace{0ptplus1cm}
学生番号を左にマークし、下に氏名を記入してください。

\vspace{3ex}

\hfill\namefield{
  \fbox{
    \begin{minipage}{.9\linewidth}
      氏名\vspace*{.5cm}%\dotfill\vspace*{.5cm}\dotfill
      \vspace*{1mm}
    \end{minipage}
  }
}
\hfill\vspace{5ex}
\end{minipage}
\hspace*{\fill}
}
% \multiSymbole{}の記号のある設問の正解は1個とは限りません。 0個の場合や複数の場合があります。

解答欄上部の「誤部正」は採点欄ですので,決して記入しないでください.

\vspace{1ex}
\hrulefill
\vspace{1ex}

%%% ヘッダー終了 %%%
%%%%%%%%%%%%%%%%%%%%%%%%%%%%%%%%%%%%%%%%%%%%%%%%%%%%%%%%%%%%%
\begin{question}{svode01}
\luaexec{
  a=math.random(2, 9);
  f='t^'..a..'*x*('..a..'+1)';
  formula=execMaxima(f);
}
微分方程式 \(\displaystyle\frac{dx}{dt} = \var{formula}\) の一般解を求めなさい.

\AMCOpen{lines=4, dots=false, framerule=0pt}{\wrongchoice[W]{誤}\scoring{0}\wrongchoice[P]{部}\scoring{1}\correctchoice[C]{正}\scoring{2}}
\end{question}
%%%%%%%%%%%%%%%%%%%%%%%%%%%%%%%%%%%%%%%%%%%%%%%%%%%%%%%%%%%%%
\begin{question}{svode02}
\luaexec{
  a=math.random(2, 9);
  f='(t+'..a..')*x^2';
  formula=execMaxima(f);
}
微分方程式 \(\displaystyle\frac{dx}{dt} = \var{formula}\) の一般解を求めなさい.

\AMCOpen{lines=4, dots=false, framerule=0pt}{\wrongchoice[W]{誤}\scoring{0}\wrongchoice[P]{部}\scoring{1}\correctchoice[C]{正}\scoring{2}}
\end{question}
%%%%%%%%%%%%%%%%%%%%%%%%%%%%%%%%%%%%%%%%%%%%%%%%%%%%%%%%%%%%%

} % End of \onecopy{}
%%%%%%%%%%%%%%%%%%%%%%%%%%%%%%%%%%%%%%%%%%%%%%%%%%%%%%%%%%%%%
\end{document}
