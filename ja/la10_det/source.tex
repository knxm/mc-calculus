\documentclass[a4paper]{ltjsarticle}
\usepackage{amsmath, amssymb}
\usepackage{fp, tikz, luacode}
\usepackage[box,completemulti,lang=JA,noshuffle]{automultiplechoice}
\newcommand*{\var}[1]{\luaexec{tex.print(#1)}}
\DeclareMathOperator{\sgn}{sgn}
\begin{document}
\AMCboxDimensions{shape=oval,width=1.8ex,height=2.5ex}
\AMCcodeVspace=0em
\luaexec{math.randomseed(2019627)} % \onecopy{} の外に配置

%\begin{luacode*}
%function execMaxima(command)
%  local maxima_command="echo 'tex1("..command..");' | maxima --very-quiet "
%  local handle=io.popen(maxima_command, "r")
%  local content=string.gsub(handle:read("*all"), "\n", "")
%  handle:close()
%  return content
%end
%function execSage(command)
%  local sage_command="sage -c 'print(latex("..command.."))'"
%  local handle=io.popen(sage_command, "r")
%  local content=string.gsub(handle:read("*all"), "\n", "")
%  handle:close()
%  return content
%end
%\end{luacode*}
%
%
% det 1digit
%
\element{det1_3}{
\begin{questionmultx}{det301}
行列式
\(
\begin{vmatrix}
 1 & 4 & 3\\
 0 & -2 & 5\\
 1 & 0 & 9
\end{vmatrix}
\) 
を求めなさい.
\FPeval\Va{8}
\AMCnumericChoices{\Va}{digits=1,sign=false,borderwidth=0pt,backgroundcol=lightgray}
\end{questionmultx}
}
\element{det1_3}{
\begin{questionmultx}{det302}
行列式
\(
\begin{vmatrix}
 0 & 1 & 1\\
 -1 & 1 & 0\\
 -3 & -4 & -3
\end{vmatrix}
\) 
を求めなさい.
\FPeval\Va{4}
\AMCnumericChoices{\Va}{digits=1,sign=false,borderwidth=0pt,backgroundcol=lightgray}
\end{questionmultx}
}
\element{det1_3}{
\begin{questionmultx}{det303}
行列式
\(
\begin{vmatrix}
 1 & 0 & -1\\
 1 & 1 & -4\\
 -2 & 0 & 9
\end{vmatrix}
\) 
を求めなさい.
\FPeval\Va{7}
\AMCnumericChoices{\Va}{digits=1,sign=false,borderwidth=0pt,backgroundcol=lightgray}
\end{questionmultx}
}
\element{det1_3}{
\begin{questionmultx}{det304}
行列式
\(
\begin{vmatrix}
 1 & 2 & 3\\
 -2 & 0 & 4\\
 0 & -1 & -1
\end{vmatrix}
\) 
を求めなさい.
\FPeval\Va{6}
\AMCnumericChoices{\Va}{digits=1,sign=false,borderwidth=0pt,backgroundcol=lightgray}
\end{questionmultx}
}
%
% det = 0
%
\element{det00_4}{
\begin{questionmultx}{det401}
行列式
\(
\begin{vmatrix}
  1 & 0 & 14 & 0\\
  1 & 1 & -1 & 1\\
  0 & -1 & -3 & -1\\
  3 & -1 & 1 & -1 
\end{vmatrix}\)
を求めなさい.
\FPeval\Va{0}
\AMCnumericChoices{\Va}{digits=1,sign=false,borderwidth=0pt,backgroundcol=lightgray}
\end{questionmultx}
}
\element{det00_4}{
\begin{questionmultx}{det402}
行列式
\(
\begin{vmatrix}
  0 & 0 & 3 & 0\\
  6 & -1 & 1 & -2\\
  0 & 0 & 12 & 0\\
  -2 & 0 & -5 & 4 
\end{vmatrix}\)
を求めなさい.
\FPeval\Va{0}
\AMCnumericChoices{\Va}{digits=1,sign=false,borderwidth=0pt,backgroundcol=lightgray}
\end{questionmultx}
}
\element{det00_4}{
\begin{questionmultx}{det403}
行列式
\(
\begin{vmatrix}
  1 & 0 & 3 & 0\\
  0 & -1 & 1 & 2\\
  6 & 1 & 8 & -2\\
  -2 & 0 & -5 & 0 
\end{vmatrix}\)
を求めなさい.
\FPeval\Va{0}
\AMCnumericChoices{\Va}{digits=1,sign=false,borderwidth=0pt,backgroundcol=lightgray}
\end{questionmultx}
}
\element{det1_4}{
\begin{questionmultx}{det101}
以下の行列式を求めなさい.計算過程を記述すること.\\
\(
\begin{vmatrix}
  3 & 0 & -1 & 0\\
  -2 & 0 & 2 & 0\\
  9 & 0 & -2 & 1\\
  1 & 1 & -1 & -1 
\end{vmatrix}
\)
\AMCOpen{lines=2, dots=false, framerule=0pt}{\wrongchoice[W]{誤}\scoring{0}\wrongchoice[P]{部}\scoring{1}\correctchoice[C]{正}\scoring{2}}
\end{questionmultx}
}
\element{det1_4}{
\begin{questionmultx}{det102}
以下の行列式を求めなさい.計算過程を記述すること.\\
\(
\begin{vmatrix}
  1 & -7 & -3 & 5\\
  0 & 1 & 1 & -1\\
  0 & -1 & 1 & 0\\
  2 & 1 & 1 & 0 
\end{vmatrix}\)
\AMCOpen{lines=2, dots=false, framerule=0pt}{\wrongchoice[W]{誤}\scoring{0}\wrongchoice[P]{部}\scoring{1}\correctchoice[C]{正}\scoring{2}}
\end{questionmultx}
}
\element{det1_4}{
\begin{questionmultx}{det103}
以下の行列式を求めなさい.過程を記述すること.\\
\(
\begin{vmatrix}
  1 & -1 & -4 & 0\\
  1 & -1 & -1 & 1\\
  -1 & 1 & 0 & 0\\
  1 & 0 & -1 & 0 
\end{vmatrix}\)
\AMCOpen{lines=2, dots=false, framerule=0pt}{\wrongchoice[W]{誤}\scoring{0}\wrongchoice[P]{部}\scoring{1}\correctchoice[C]{正}\scoring{2}}
\end{questionmultx}
}
\element{det1_4}{
\begin{questionmultx}{det104}
以下の行列式を求めなさい.過程を記述すること.\\
\(
\begin{vmatrix}
  1 & 0 & 0 & 0\\
  9 & 1 & 0 & 0\\
  3 & 8 & 1 & -2\\
  -1 & 1 & 0 & 3 
\end{vmatrix}\)
% and 3 
\AMCOpen{lines=2, dots=false, framerule=0pt}{\wrongchoice[W]{誤}\scoring{0}\wrongchoice[P]{部}\scoring{1}\correctchoice[C]{正}\scoring{2}}
\end{questionmultx}
}
\element{det2_4}{
\begin{questionmultx}{det201}
行列式
\(
\begin{vmatrix}
  -1 & 2 & -7 & 1\\
  2 & 1 & 3 & -1\\
  -1 & -1 & 0 & 3\\
  1 & 1 & 2 & 2 
\end{vmatrix}\)
を求めなさい.
\FPeval\Va{16}
\AMCnumericChoices{\Va}{digits=2,sign=true,borderwidth=0pt,backgroundcol=lightgray}
\end{questionmultx}
}
\element{det2_4}{
\begin{questionmultx}{det202}
行列式
\(
\begin{vmatrix}
  -2 & 9 & 1 & 5\\
  -1 & -2 & 0 & 0\\
  2 & 2 & -1 & 0\\
  1 & -2 & -3 & -1 
\end{vmatrix}\)
を求めなさい.
\FPeval\Va{21}
\AMCnumericChoices{\Va}{digits=2,sign=true,borderwidth=0pt,backgroundcol=lightgray}
\end{questionmultx}
}
\element{det2_4}{
\begin{questionmultx}{det203}
行列式
\(
\begin{vmatrix}
  0 & 5 & -2 & 1\\
  -2 & 2 & -1 & 0\\
  -2 & 6 & 0 & -1\\
  1 & 3 & 1 & -2 
\end{vmatrix}\)
を求めなさい.
\FPeval\Va{35*(-1)}
\AMCnumericChoices{\Va}{digits=2,sign=true,borderwidth=0pt,backgroundcol=lightgray}
\end{questionmultx}
}
\element{det2_4}{
\begin{questionmultx}{det204}
行列式
\(
\begin{vmatrix}
  1 & -1 & -1 & -1\\
  -1 & 1 & -1 & 0\\
  3 & 5 & -1 & -1\\ 
  -1 & -2 & 5 & 1
\end{vmatrix}\)
を求めなさい.
\FPeval\Va{26*(-1)}
\AMCnumericChoices{\Va}{digits=2,sign=true,borderwidth=0pt,backgroundcol=lightgray}
\end{questionmultx}
}

\onecopy{2}{
%%% ヘッダー開始: %%%   
\noindent{\textbf{線形代数 演習10 \hfill 2019年6月27日}}

\vspace{3ex}

{\small
  \setlength{\parindent}{0pt}\hspace*{\fill}\AMCcodeGridInt{学生番号}{8}\hspace*{\fill}
\begin{minipage}[b]{6.5cm}
$\longleftarrow{}$\hspace{0ptplus1cm}
学生番号を左にマークし,下に氏名を記入してください.
%NU-AppsGのメールアドレス(brxxyyzzz)
%xxは名前から,yyは入学年度,zzzは学生番号下3桁です.@以降は必要ありません.

\vspace{3ex}

\hfill\namefield{
  \fbox{
    \begin{minipage}{.9\linewidth}
      氏名\vspace*{.5cm}%\dotfill\vspace*{.5cm}\dotfill
      \vspace*{1mm}
    \end{minipage}
  }
}
\hfill\vspace{5ex}
\end{minipage}
\hspace*{\fill}
}
% \multiSymbole{}の記号のある設問の正解は1個とは限りません。 0個の場合や複数の場合があります。

% 解答欄上部の「誤部正」は採点欄ですので,決して記入しないでください.
解答欄に符号\((+, -)\)がある場合,正の数は\(+\),負の数は\(-\)を忘れずにマークすること.
また,解答の数値が2桁の場合は,上が「十の位」,下が「一の位」です.

\vspace{1ex}
\hrulefill
\vspace{1ex}

%%% ヘッダー終了 %%%
%%%%%%%%%%%%%%%%%%%%%%%%%%%%%%%%%%%%%%%%%%%%%%%%%%%%%%%%%%%%%
\shufflegroup{det1_3}
\insertgroup[1]{det1_3}

\shufflegroup{det2_4}
\insertgroup[1]{det2_4}

\shufflegroup{det1_4}
\insertgroup[1]{det1_4}

} %end of \onecopy{}
\end{document}

