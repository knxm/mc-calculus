\documentclass[a4paper]{ltjsarticle}
\usepackage{amsmath, amssymb}
\usepackage{tikz}
\usepackage{luacode}
\usepackage[box,completemulti,lang=JA]{automultiplechoice}
\newcommand*{\var}[1]{\luaexec{tex.print(#1)}}
\begin{document}
\AMCboxDimensions{shape=oval,width=1.8ex,height=2.5ex}
\luadirect{math.randomseed(190515)} % \onecopy{} の外に配置

\begin{luacode*}
function execSage(command)
  local sage_command="sage -c 'print(latex("..command.."))'"
  local handle=io.popen(sage_command, "r")
  local content=string.gsub(handle:read("*all"), "\n", "")
  handle:close()
  return content
end
function execMaxima(command)
  local maxima_command="echo 'tex1("..command..");' | maxima --very-quiet "
  local handle=io.popen(maxima_command, "r")
  local content=string.gsub(handle:read("*all"), "\n", "")
  handle:close()
  return content
end
\end{luacode*}

\onecopy{2}{

%%% ヘッダー開始: %%%   
\noindent{\bf 応用数学 演習06 \hfill 2019年5月15日}

\vspace{3ex}

{\small
  \setlength{\parindent}{0pt}\hspace*{\fill}\AMCcodeGridInt{学生番号}{8}\hspace*{\fill}
\begin{minipage}[b]{6.5cm}$\longleftarrow{}$\hspace{0ptplus1cm}
学生番号を左にマークし、下に氏名を記入してください。

\vspace{3ex}

\hfill\namefield{
  \fbox{
    \begin{minipage}{.9\linewidth}
      氏名\vspace*{.5cm}%\dotfill\vspace*{.5cm}\dotfill
      \vspace*{1mm}
    \end{minipage}
  }
}
\hfill\vspace{5ex}
\end{minipage}
\hspace*{\fill}
}

\vspace{1ex}
%計算は自分のノートに書くこと.答案用紙に書いてはいけません.
%\multiSymbole{}の記号のある設問の正解は1個とは限りません。 0個の場合や複数の場合があります。

\vspace{1ex}
\hrulefill
\vspace{1ex}

%%% ヘッダー終了 %%%
%%%%%%%%%%%%%%%%%%%%%%%%%%%%%%%%%%%%%%%%%%%%%%%%%%%%%%%%%%%%%
\begin{question}{int01}
\luaexec{
  a=math.random(3, 9)
  f='1/x^'..a
  formula=execMaxima(f)
  correct=execMaxima('integrate('..f..', x)')
  wrong1=formula
  wrong2=execMaxima('diff('..f..', x)')
  wrong3=execMaxima('integrate('..f..'*('..a..'-1), x)')
  wrong4=execMaxima('-1*integrate('..f..', x)')
}
不定積分$\displaystyle\int \var{formula}\, dx$を求めなさい.
ただし,積分定数$C$は省略している.
  \begin{choiceshoriz}
    \correctchoice{\(\var{correct}\)}
    \wrongchoice{\(\var{wrong1}\)}
    \wrongchoice{\(\var{wrong2}\)}
    \wrongchoice{\(\var{wrong3}\)}
    \wrongchoice{\(\var{wrong4}\)}
  \end{choiceshoriz}
\end{question}

%%%%%%%%%%%%%%%%%%%%%%%%%%%%%%%%%%%%%%%%%%%%%%%%%%%%%%%%%%%%%
\begin{question}{int02}
\luaexec{
  a=math.random(2, 9)
  f='e^{'..a..'x}'
  formula=execMaxima(f)
  correct='\\frac{'..f..'}{'..a..'}'
%  correct=execMaxima('integrate('..f..', x)')
  wrong1=f
  wrong2=a..'e^{'..a..'x}'
  wrong3='\\frac{'..f..'}{'..a..'x}'
  wrong4='\\frac{e^{'..a..'x+1}}{'..a..'x}'
}
不定積分$\displaystyle\int \var{f} dx$を求めなさい.
ただし,積分定数$C$は省略している.
  \begin{choiceshoriz}
    \correctchoice{\(\var{correct}\)}
    \wrongchoice{\(\var{wrong1}\)}
    \wrongchoice{\(\var{wrong2}\)}
    \wrongchoice{\(\var{wrong3}\)}
    \wrongchoice{\(\var{wrong4}\)}
  \end{choiceshoriz}
\end{question}

%%%%%%%%%%%%%%%%%%%%%%%%%%%%%%%%%%%%%%%%%%%%%%%%%%%%%%%%%%%%%
\begin{question}{int03}
\luaexec{
  a=math.random(2, 5)
  table={7, 11, 13}
  i=math.random(1, 3)
  f='e^{-'..a..'x+'..table[i]..'}'
  correct='-\\frac{'..f..'}{'..a..'}'
  wrong1=f
  wrong2=-a..'e^{-'..a..'x+'..table[i]..'}'
  wrong3='\\frac{'..f..'}{-'..a..'x+'..table[i]..'}'
}
不定積分$\displaystyle\int \var{f} dx$を求めなさい.
ただし,積分定数$C$は省略している.
  \begin{choiceshoriz}
    \correctchoice{\(\var{correct}\)}
    \wrongchoice{\(\var{wrong1}\)}
    \wrongchoice{\(\var{wrong2}\)}
    \wrongchoice{\(\var{wrong3}\)}
  \end{choiceshoriz}
\end{question}

%%%%%%%%%%%%%%%%%%%%%%%%%%%%%%%%%%%%%%%%%%%%%%%%%%%%%%%%%%%%%
\begin{question}{int04}
\luaexec{
  table={5, 7, 11, 13}
  i=math.random(1, 4)
  a=table[i]
  b=math.random(1, 4)
  g=a..'*x-'..b
  f='1/('..g..')'
  formula=execMaxima(f)
  correct='{\\log \\left|'..a..'\\,x-'..b..'\\right|}\\over{'..a..'}'
  wrong1='\\log \\left|'..a..'\\,x-'..b..'\\right|'
  wrong2=execMaxima('diff('..f..', x)')
  wrong3=execMaxima('integrate('..f..', x)')
}
不定積分$\displaystyle\int \var{formula}\, dx$を求めなさい.
ただし,積分定数$C$は省略している.
  \begin{choiceshoriz}
    \correctchoice{\(\var{correct}\)}
    \wrongchoice{\(\var{wrong1}\)}
    \wrongchoice{\(\var{wrong2}\)}
    \wrongchoice{\(\var{wrong3}\)}
  \end{choiceshoriz}
\end{question}

%%%%%%%%%%%%%%%%%%%%%%%%%%%%%%%%%%%%%%%%%%%%%%%%%%%%%%%%%%%%%
\begin{question}{int05}
\luaexec{
  a=math.random(2, 5)
  g='sqrt('..a..'^2-x^2)'
  gformula=execMaxima(g)
  f='1/('..g..')'
  formula=execMaxima(f)
  correct=execMaxima('integrate('..f..', x)')
  wrong1='\\log'..gformula
  wrong2=execMaxima('diff('..f..', x)')
  wrong3=execMaxima(a..'*integrate('..f..', x)')
}
不定積分$\displaystyle\int \var{formula}\, dx$を求めなさい.
ただし,積分定数$C$は省略している.
  \begin{choiceshoriz}
    \correctchoice{\(\var{correct}\)}
    \wrongchoice{\(\var{wrong1}\)}
    \wrongchoice{\(\var{wrong2}\)}
    \wrongchoice{\(\var{wrong3}\)}
  \end{choiceshoriz}
\end{question}
%%%%%%%%%%%%%%%%%%%%%%%%%%%%%%%%%%%%%%%%%%%%%%%%%%%%%%%%%%%%%
\begin{question}{int06}
\luaexec{
  a=math.random(3, 9)
  b=math.random(3, 9)
  f='cos('..a..'*x+'..b..')'
  formula=execMaxima(f)
  correct=execMaxima('integrate('..f..', x)')
  wrong1=formula
  wrong2=execMaxima('diff('..f..', x)')
  wrong3=execMaxima('-1*integrate('..f..', x)')
}
不定積分$\displaystyle\int \var{formula}\, dx$を求めなさい.
ただし,積分定数$C$は省略している.
  \begin{choiceshoriz}
    \correctchoice{\(\var{correct}\)}
    \wrongchoice{\(\var{wrong1}\)}
    \wrongchoice{\(\var{wrong2}\)}
    \wrongchoice{\(\var{wrong3}\)}
  \end{choiceshoriz}
\end{question}
%%%%%%%%%%%%%%%%%%%%%%%%%%%%%%%%%%%%%%%%%%%%%%%%%%%%%%%%%%%%%
\begin{question}{int07}
\luaexec{
  a=math.random(3, 9)
  b=math.random(3, 9)
  f='sin('..a..'*x+'..b..')'
  formula=execMaxima(f)
  correct=execMaxima('integrate('..f..', x)')
  wrong1=formula
  wrong2=execMaxima('diff('..f..', x)')
  wrong3=execMaxima('-1*integrate('..f..', x)')
}
不定積分$\displaystyle\int \var{formula}\, dx$を求めなさい.
ただし,積分定数$C$は省略している.
  \begin{choiceshoriz}
    \correctchoice{\(\var{correct}\)}
    \wrongchoice{\(\var{wrong1}\)}
    \wrongchoice{\(\var{wrong2}\)}
    \wrongchoice{\(\var{wrong3}\)}
  \end{choiceshoriz}
\end{question}

} % end of \onecopy{}
%%%%%%%%%%%%%%%%%%%%%%%%%%%%%%%%%%%%%%%%%%%%%%%%%%%%%%%%%%%%%
\end{document}