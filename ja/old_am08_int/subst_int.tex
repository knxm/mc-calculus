%%%%%%%%%%%%%%%%%%%%%%%%%%%%%%%%%%%%%%%%%%%%%%%%%%%%%%%%%%%%%
\element{subst_int}{
\begin{question}{sint01}
\luaexec{
  a=math.random(4, 9);
  f='sin(x)^('..a..')*cos(x)';
  formula=execMaxima(f);
  correct=execMaxima('integrate('..f..', x)');
  wrong1=formula;
  wrong2=execMaxima('diff('..f..', x)');
  wrong3=execMaxima('integrate('..f..'*('..a..'-1), x)');
  wrong4=execMaxima('-1*integrate('..f..', x)');
}
不定積分$\displaystyle\int \var{formula}\, dx$を求めなさい.
ただし,積分定数$C$は省略している.
  \begin{choiceshoriz}
    \correctchoice{\(\var{correct}\)}
    \wrongchoice{\(\var{wrong1}\)}
    \wrongchoice{\(\var{wrong2}\)}
    \wrongchoice{\(\var{wrong3}\)}
    \wrongchoice{\(\var{wrong4}\)}
  \end{choiceshoriz}
\end{question}}
%%%%%%%%%%%%%%%%%%%%%%%%%%%%%%%%%%%%%%%%%%%%%%%%%%%%%%%%%%%%%
\element{subst_int}{
\begin{question}{sint02}
\luaexec{
  table={1, 3, 5, 7};
  i=math.random(1, 4);
  a=table[i];
  b=math.random(1, 4);
  c=math.random(3, 7);
  g='x^2+'..a..'*x+'..b;
  f='(2*x+'..a..')*(x^2+'..a..'*x+'..b..')^'..c;
  formula=execMaxima(f);
  correct=execMaxima('integrate('..f..', x)');
  wrong1=formula;
  wrong2=execSage('diff('..f..', x)'); 
  wrong3=execMaxima('('..c..'+1)*integrate('..f..', x)');
  wrong4=execMaxima('-1*integrate('..f..', x)');
}
不定積分$\displaystyle\int\var{formula}\, dx$を求めなさい.
ただし,積分定数$C$は省略している.
  \begin{choiceshoriz}
    \correctchoice{\(\var{correct}\)}
    \wrongchoice{\(\var{wrong1}\)}
    \wrongchoice{\(\var{wrong2}\)}
    \wrongchoice{\(\var{wrong3}\)}
    \wrongchoice{\(\var{wrong4}\)}
  \end{choiceshoriz}
\end{question}
}
%%%%%%%%%%%%%%%%%%%%%%%%%%%%%%%%%%%%%%%%%%%%%%%%%%%%%%%%%%%%%
\element{subst_int}{
\begin{question}{sint03}
\luaexec{
  a=math.random(2, 5);
  b=math.random(1, 9);
  f='exp('..a..'*x)/(exp('..a..'*x)+'..b..')';
  formula=execMaxima(f);
  correct=execMaxima('integrate('..f..', x)');
  wrong1=formula;
  wrong2=execMaxima('exp('..a..'*x)/('..a..'*(exp('..a..'*x)+'..b..'))');
  wrong3=execMaxima(a..'*integrate('..f..', x)');
  wrong4=execMaxima('-integrate('..f..', x)');
}
不定積分$\displaystyle\int\var{formula}\, dx$を求めなさい.
ただし,積分定数$C$は省略している.
  \begin{choiceshoriz}
    \correctchoice{\(\var{correct}\)}
    \wrongchoice{\(\var{wrong1}\)}
    \wrongchoice{\(\var{wrong2}\)}
    \wrongchoice{\(\var{wrong3}\)}
    \wrongchoice{\(\var{wrong4}\)}
  \end{choiceshoriz}
\end{question}
}
%%%%%%%%%%%%%%%%%%%%%%%%%%%%%%%%%%%%%%%%%%%%%%%%%%%%%%%%%%%%%
\element{subst_int}{
\begin{question}{sint04}
\luaexec{
  a=math.random(2, 5);
  b=math.random(1, 9);
  f='log('..a..'*x+'..b..')/('..a..'*x+'..b..')';
  formula=execMaxima(f);
  correct=execMaxima('integrate('..f..', x)');
  wrong1=formula;
  wrong2=execMaxima(a..'*integrate('..f..', x)');
  wrong3=execMaxima('2*'..a..'*integrate('..f..', x)');
  wrong4=execMaxima('-integrate('..f..', x)');
}
不定積分$\displaystyle\int\var{formula}\, dx$を求めなさい.
ただし,積分定数$C$は省略している.
  \begin{choiceshoriz}
    \correctchoice{\(\var{correct}\)}
    \wrongchoice{\(\var{wrong1}\)}
    \wrongchoice{\(\var{wrong2}\)}
    \wrongchoice{\(\var{wrong3}\)}
    \wrongchoice{\(\var{wrong4}\)}
  \end{choiceshoriz}
\end{question}
}
%%%%%%%%%%%%%%%%%%%%%%%%%%%%%%%%%%%%%%%%%%%%%%%%%%%%%%%%%%%%%
\element{subst_int}{
\begin{question}{sint05}
\luaexec{
  a=math.random(2, 7);
  f='x/sqrt('..a..'-x)';
  formula=execMaxima(f);
  correct=execMaxima('integrate('..f..', x)');
  wrong1=formula;
  wrong2=execMaxima('ratsimp(3*integrate('..f..', x))');
  wrong3=execMaxima('asin(x/'..a..')');
  wrong4=execMaxima('-integrate('..f..', x)');
}
不定積分$\displaystyle\int\var{formula}\, dx$を求めなさい.
ただし,積分定数$C$は省略している.
  \begin{choiceshoriz}
    \correctchoice{\(\var{correct}\)}
    \wrongchoice{\(\var{wrong1}\)}
    \wrongchoice{\(\var{wrong2}\)}
    \wrongchoice{\(\var{wrong3}\)}
    \wrongchoice{\(\var{wrong4}\)}
  \end{choiceshoriz}
\end{question}
}
%%%%%%%%%%%%%%%%%%%%%%%%%%%%%%%%%%%%%%%%%%%%%%%%%%%%%%%%%%%%%
\element{subst_int}{
\begin{question}{sint06}
\luaexec{
  a=math.random(2, 7);
  b=math.random(2, 7);
  f='x*(x-'..a..')^'..b;
  formula=execMaxima(f);
  correct=execMaxima('(x-'..a..')^('..b..'+2)/('..b..'+2)+'..a..'*(x-'..a..')^('..b..'+1)/('..b..'+1)');
  wrong1=execMaxima('(x-'..a..')^('..b..'+1)/('..b..'+1)+'..a..'*(x-'..a..')^('..b..')/('..b..')');
  wrong2=execMaxima('(x-'..a..')^('..b..'+1)/('..b..'+1)');
  wrong3=execMaxima('diff('..f..', x)')
  wrong4=formula;
 }
不定積分$\displaystyle\int\var{formula}\, dx$を求めなさい.
ただし,積分定数$C$は省略している.
  \begin{choiceshoriz}
    \correctchoice{\(\var{correct}\)}
    \wrongchoice{\(\var{wrong1}\)}
    \wrongchoice{\(\var{wrong2}\)}
    \wrongchoice{\(\var{wrong3}\)}
    \wrongchoice{\(\var{wrong4}\)}
  \end{choiceshoriz}
\end{question}
}

\element{subst_int}{
\begin{question}{sint07}
\luaexec{
  a=math.random(2, 9);
  b=math.random(0, 1);
  c=(-1)^b*a;
  f='x/(x+'..c..')^2';
  formula=execMaxima(f);
  correct=execMaxima('log(abs(x+'..c..'))+2/(x+'..c..')');
  wrong1=execMaxima('log(x+'..c..')+2/(x+'..c..')');
  wrong2=execMaxima('1/(x+'..c..')');
  wrong3=execMaxima('-1/(x+'..c..')');
  wrong4=execMaxima('2*log(abs(x+'..c..'))');
 }
不定積分$\displaystyle\int\var{formula}\, dx$を求めなさい.
ただし,積分定数$C$は省略している.
  \begin{choiceshoriz}
    \correctchoice{\(\var{correct}\)}
    \wrongchoice{\(\var{wrong1}\)}
    \wrongchoice{\(\var{wrong2}\)}
    \wrongchoice{\(\var{wrong3}\)}
    \wrongchoice{\(\var{wrong4}\)}
  \end{choiceshoriz}
\end{question}
}
