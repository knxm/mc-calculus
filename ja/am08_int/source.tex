\documentclass[a4paper]{ltjsarticle}
\usepackage{amsmath, amssymb}
\usepackage{tikz}
\usepackage{luacode}
\usepackage[box,completemulti,lang=JA]{automultiplechoice}
\newcommand*{\var}[1]{\luaexec{tex.print(#1)}}
\begin{document}
\AMCboxDimensions{shape=oval,width=1.8ex,height=2.5ex}
\luaexec{math.randomseed(190529)} % \onecopy{} の外に配置

\begin{luacode*}
function execMaxima(command)
  local maxima_command="echo 'tex1("..command..");' | maxima --very-quiet "
  local handle=io.popen(maxima_command, "r")
  local content=string.gsub(handle:read("*all"), "\n", "")
  handle:close()
  return content
end
function execSage(command)
  local sage_command="sage -c 'print(latex("..command.."))'"
  local handle=io.popen(sage_command, "r")
  local content=string.gsub(handle:read("*all"), "\n", "")
  handle:close()
  return content
end
\end{luacode*}

%%%%%%%%%%%%%%%%%%%%%%%%%%%%%%%%%%%%%%%%%%%%%%%%%%%%%%%%%%%%%
\element{subst_int}{
\begin{question}{sint01}
\luaexec{
  a=math.random(4, 9);
  f='sin(x)^('..a..')*cos(x)';
  formula=execMaxima(f);
}
不定積分$\displaystyle\int \var{formula}\, dx$を求めなさい.
\AMCOpen{lines=4, dots=false, framerule=0pt}{\wrongchoice[W]{誤}\scoring{0}\wrongchoice[P]{部}\scoring{1}\correctchoice[C]{正}\scoring{2}}
\end{question}}
%%%%%%%%%%%%%%%%%%%%%%%%%%%%%%%%%%%%%%%%%%%%%%%%%%%%%%%%%%%%%
\element{subst_int}{
\begin{question}{sint02}
\luaexec{
  table={1, 3, 5, 7};
  i=math.random(1, 4);
  a=table[i];
  b=math.random(1, 4);
  c=math.random(3, 7);
  g='x^2+'..a..'*x+'..b;
  f='(2*x+'..a..')*(x^2+'..a..'*x+'..b..')^'..c;
  formula=execMaxima(f);
}
不定積分$\displaystyle\int\var{formula}\, dx$を求めなさい.
\AMCOpen{lines=4, dots=false, framerule=0pt}{\wrongchoice[W]{誤}\scoring{0}\wrongchoice[P]{部}\scoring{1}\correctchoice[C]{正}\scoring{2}}
\end{question}
}
%%%%%%%%%%%%%%%%%%%%%%%%%%%%%%%%%%%%%%%%%%%%%%%%%%%%%%%%%%%%%
\element{subst_int}{
\begin{question}{sint03}
\luaexec{
  a=math.random(2, 5);
  b=math.random(1, 9);
  f='exp('..a..'*x)/(exp('..a..'*x)+'..b..')';
  formula=execMaxima(f);
}
不定積分$\displaystyle\int\var{formula}\, dx$を求めなさい.
\AMCOpen{lines=4, dots=false, framerule=0pt}{\wrongchoice[W]{誤}\scoring{0}\wrongchoice[P]{部}\scoring{1}\correctchoice[C]{正}\scoring{2}}
\end{question}
}
%%%%%%%%%%%%%%%%%%%%%%%%%%%%%%%%%%%%%%%%%%%%%%%%%%%%%%%%%%%%%
\element{subst_int}{
\begin{question}{sint04}
\luaexec{
  a=math.random(2, 5);
  b=math.random(1, 9);
  f='log('..a..'*x+'..b..')/('..a..'*x+'..b..')';
  formula=execMaxima(f);
}
不定積分$\displaystyle\int\var{formula}\, dx$を求めなさい.
\AMCOpen{lines=4, dots=false, framerule=0pt}{\wrongchoice[W]{誤}\scoring{0}\wrongchoice[P]{部}\scoring{1}\correctchoice[C]{正}\scoring{2}}
\end{question}
}
%%%%%%%%%%%%%%%%%%%%%%%%%%%%%%%%%%%%%%%%%%%%%%%%%%%%%%%%%%%%%
\element{subst_int}{
\begin{question}{sint05}
\luaexec{
  a=math.random(2, 7);
  f='x/sqrt('..a..'-x)';
  formula=execMaxima(f);
}
不定積分$\displaystyle\int\var{formula}\, dx$を求めなさい.
\AMCOpen{lines=4, dots=false, framerule=0pt}{\wrongchoice[W]{誤}\scoring{0}\wrongchoice[P]{部}\scoring{1}\correctchoice[C]{正}\scoring{2}}
\end{question}
}
%%%%%%%%%%%%%%%%%%%%%%%%%%%%%%%%%%%%%%%%%%%%%%%%%%%%%%%%%%%%%
\element{subst_int}{
\begin{question}{sint06}
\luaexec{
  a=math.random(2, 7);
  b=math.random(2, 7);
  f='x*(x-'..a..')^'..b;
  formula=execMaxima(f);
}
不定積分$\displaystyle\int\var{formula}\, dx$を求めなさい.
\AMCOpen{lines=4, dots=false, framerule=0pt}{\wrongchoice[W]{誤}\scoring{0}\wrongchoice[P]{部}\scoring{1}\correctchoice[C]{正}\scoring{2}}
\end{question}
}

\element{subst_int}{
\begin{question}{sint07}
\luaexec{
  a=math.random(2, 9);
  b=math.random(0, 1);
  c=(-1)^b*a;
  f='x/(x+'..c..')^2';
  formula=execMaxima(f);
}
不定積分$\displaystyle\int\var{formula}\, dx$を求めなさい.
\AMCOpen{lines=4, dots=false, framerule=0pt}{\wrongchoice[W]{誤}\scoring{0}\wrongchoice[P]{部}\scoring{1}\correctchoice[C]{正}\scoring{2}}
\end{question}
}

%%%%%%%%%%%%%%%%%%%%%%%%%%%%%%%%%%%%%%%%%%%%%%%%%%%%%%%%%%%%%
\element{partial_int}{
\begin{question}{pint01}
\luaexec{
  a=math.random(2, 7);
  f='x*exp('..a..'*x)';
  formula=execMaxima(f);
  correct=execMaxima('integrate('..f..', x)');
  wrong1=formula;
  wrong2=execMaxima(a..'^2*integrate('..f..', x)');
  wrong3=execMaxima('x*exp('..a..'*x)/'..a);
  wrong4=execMaxima('ratsimp(integrate('..f..', x)+2*exp('..a..'*x)/'..a..'^2)');
}
不定積分$\displaystyle\int\var{formula}\, dx$を求めなさい.
ただし,積分定数$C$は省略している.
  \begin{choiceshoriz}
    \correctchoice{\(\var{correct}\)}
    \wrongchoice{\(\var{wrong1}\)}
    \wrongchoice{\(\var{wrong2}\)}
    \wrongchoice{\(\var{wrong3}\)}
    \wrongchoice{\(\var{wrong4}\)}
  \end{choiceshoriz}
\end{question}
}

%%%%%%%%%%%%%%%%%%%%%%%%%%%%%%%%%%%%%%%%%%%%%%%%%%%%%%%%%%%%%
\element{partial_int}{
\begin{question}{pint02}
\luaexec{
  a=math.random(2, 9);
  table={'sin', 'cos'};
  i=math.random(1, 2);
  if i==1 then j=2 else j=1 end;
  trig1=table[i];
  trig2=table[j];
  f='x*'..trig1..'('..a..'*x)';
  formula=execMaxima(f);
  correct=execMaxima('integrate('..f..',x)');
  wrong1=execMaxima('x*sin('..a..'*x)/'..a);
  wrong2=execMaxima('-x*cos('..a..'*x)/'..a);
  wrong3=execMaxima('('..a..'*x*'..trig1..'('..a..'*x)+(-1)^'..i..'*'..trig2..'('..a..'*x))/'..a..'^2');
  wrong4=execMaxima('('..a..'*x*'..trig1..'('..a..'*x)+(-1)^('..i..'+1)*'..trig2..'('..a..'*x))/'..a..'^2');
 }
不定積分$\displaystyle\int\var{formula}\, dx$を求めなさい.
ただし,積分定数$C$は省略している.
  \begin{choiceshoriz}
    \correctchoice{\(\var{correct}\)}
    \wrongchoice{\(\var{wrong1}\)}
    \wrongchoice{\(\var{wrong2}\)}
    \wrongchoice{\(\var{wrong3}\)}
    \wrongchoice{\(\var{wrong4}\)}
  \end{choiceshoriz}
\end{question}
}

%%%%%%%%%%%%%%%%%%%%%%%%%%%%%%%%%%%%%%%%%%%%%%%%%%%%%%%%%%%%%
\element{partial_int}{
\begin{question}{pint03}
\luaexec{
  a=math.random(2, 9);
  f='x^'..a..'*log(x)';
  g='x^('..a..'-1)*log(x)';
  formula=execMaxima(f);
  correct=execMaxima('integrate('..f..',x)');
  wrong1=execMaxima('diff('..f..', x)');
  wrong2=execMaxima('integrate('..g..',x)');
  wrong3=execMaxima('1/('..a..'+1)*x^('..a..'+1)*log(x)+(1/('..a..'+1)^2)*x^('..a..'+1)')
  wrong4=execMaxima('1/('..a..')*x^('..a..')*log(x)+(1/('..a..')^2)*x^('..a..')')
}
不定積分$\displaystyle\int\var{formula}\, dx$を求めなさい.
ただし,積分定数$C$は省略している.
  \begin{choiceshoriz}
    \correctchoice{\(\var{correct}\)}
    \wrongchoice{\(\var{wrong1}\)}
    \wrongchoice{\(\var{wrong2}\)}
    \wrongchoice{\(\var{wrong3}\)}
    \wrongchoice{\(\var{wrong4}\)}
  \end{choiceshoriz}
\end{question}
}

%%%%%%%%%%%%%%%%%%%%%%%%%%%%%%%%%%%%%%%%%%%%%%%%%%%%%%%%%%%%%
\element{partial_int}{
\begin{question}{pint04}
\luaexec{
  f='x^2*cos(x)';
  formula=execMaxima(f);
  correct=execMaxima('integrate('..f..',x)');
  wrong1=execMaxima('diff('..f..', x)');
  wrong2=execMaxima('(x^2+2)*sin(x)-2*x*cos(x)')
  wrong3=execMaxima('(x^2+2)*sin(x)+2*x*cos(x)')
  wrong4=execMaxima('(x^2-2)*sin(x)-2*x*cos(x)')
}
不定積分$\displaystyle\int\var{formula}\, dx$を求めなさい.
ただし,積分定数$C$は省略している.
  \begin{choiceshoriz}
    \correctchoice{\(\var{correct}\)}
    \wrongchoice{\(\var{wrong1}\)}
    \wrongchoice{\(\var{wrong2}\)}
    \wrongchoice{\(\var{wrong3}\)}
    \wrongchoice{\(\var{wrong4}\)}
  \end{choiceshoriz}
\end{question}
}

%%%%%%%%%%%%%%%%%%%%%%%%%%%%%%%%%%%%%%%%%%%%%%%%%%%%%%%%%%%%%
\element{partial_int}{
\begin{question}{pint05}
\luaexec{
  f='x^2*sin(x)';
  formula=execMaxima(f);
  correct=execMaxima('integrate('..f..',x)');
  wrong1=execMaxima('diff('..f..', x)');
  wrong2=execMaxima('-2*x*sin(x)+(2-x^2)*cos(x)')
  wrong3=execMaxima('-2*x*sin(x)-(x^2+2)*cos(x)')
  wrong4=execMaxima('2*x*sin(x)+(x^2+2)*cos(x)')
}
不定積分$\displaystyle\int\var{formula}\, dx$を求めなさい.
ただし,積分定数$C$は省略している.
  \begin{choiceshoriz}
    \correctchoice{\(\var{correct}\)}
    \wrongchoice{\(\var{wrong1}\)}
    \wrongchoice{\(\var{wrong2}\)}
    \wrongchoice{\(\var{wrong3}\)}
    \wrongchoice{\(\var{wrong4}\)}
  \end{choiceshoriz}
\end{question}
}

%%%%%%%%%%%%%%%%%%%%%%%%%%%%%%%%%%%%%%%%%%%%%%%%%%%%%%%%%%%%%
\element{partial_int}{
\begin{question}{pint06}
\luaexec{
  a=math.random(2, 9);
  f='x*(log(x))^2';
  formula=execMaxima(f);
  correct=execMaxima('integrate('..f..', x)');
  wrong1=formula;
  wrong2=execMaxima('x^2*(2*(log(x))^2+2*log(x)-1)/4')
  wrong3=execMaxima('diff('..f..', x)')
  wrong4=execMaxima('x^2*(2*(log(x))^2-2*log(x)-1)/4')
}
不定積分$\displaystyle\int\var{formula}\, dx$を求めなさい.
ただし,積分定数$C$は省略している.
  \begin{choiceshoriz}
    \correctchoice{\(\var{correct}\)}
    \wrongchoice{\(\var{wrong1}\)}
    \wrongchoice{\(\var{wrong2}\)}
    \wrongchoice{\(\var{wrong3}\)}
    \wrongchoice{\(\var{wrong4}\)}
  \end{choiceshoriz}
\end{question}
}

\onecopy{2}{

%%% ヘッダー開始: %%%   
\noindent{\bf 応用数学 演習08 \hfill 2019年5月29日}

\vspace{3ex}

{\small
  \setlength{\parindent}{0pt}\hspace*{\fill}\AMCcodeGridInt{学生番号}{8}\hspace*{\fill}
\begin{minipage}[b]{6.5cm}$\longleftarrow{}$\hspace{0ptplus1cm}
学生番号を左にマークし、下に氏名を記入してください。

\vspace{3ex}

\hfill\namefield{
  \fbox{
    \begin{minipage}{.9\linewidth}
      氏名\vspace*{.5cm}%\dotfill\vspace*{.5cm}\dotfill
      \vspace*{1mm}
    \end{minipage}
  }
}
\hfill\vspace{5ex}
\end{minipage}
\hspace*{\fill}
}

% \multiSymbole{}の記号のある設問の正解は1個とは限りません。 0個の場合や複数の場合があります。

\vspace{1ex}
\hrulefill
\vspace{1ex}

%%% ヘッダー終了 %%%
%%%%%%%%%%%%%%%%%%%%%%%%%%%%%%%%%%%%%%%%%%%%%%%%%%%%%%%%%%%%%
\cleargroup{all}
\copygroup[3]{subst_int}{all}
\copygroup[3]{partial_int}{all}
\shufflegroup{all}
\insertgroup{all}

} % End of \onecopy{}
%%%%%%%%%%%%%%%%%%%%%%%%%%%%%%%%%%%%%%%%%%%%%%%%%%%%%%%%%%%%%
\end{document}
