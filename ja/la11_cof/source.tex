\documentclass[a4paper]{ltjsarticle}
\usepackage{amsmath, amssymb}
\usepackage{fp, tikz, luacode}
\usepackage[box,completemulti,lang=JA,noshuffle]{automultiplechoice}
\newcommand*{\var}[1]{\luaexec{tex.print(#1)}}
\DeclareMathOperator{\sgn}{sgn}
\begin{document}
\AMCboxDimensions{shape=oval,width=1.8ex,height=2.5ex}
\AMCcodeVspace=0em
\luaexec{math.randomseed(2019704)} % \onecopy{} の外に配置

%\begin{luacode*}
%function execMaxima(command)
%  local maxima_command="echo 'tex1("..command..");' | maxima --very-quiet "
%  local handle=io.popen(maxima_command, "r")
%  local content=string.gsub(handle:read("*all"), "\n", "")
%  handle:close()
%  return content
%end
%function execSage(command)
%  local sage_command="sage -c 'print(latex("..command.."))'"
%  local handle=io.popen(sage_command, "r")
%  local content=string.gsub(handle:read("*all"), "\n", "")
%  handle:close()
%  return content
%end
%\end{luacode*}
%
%
% cof
%
%%%%%%%%%%%%%%
\element{cof}{
\begin{question}{cof01}
行列
\(
A=\begin{pmatrix}
 1 & -2 & -1\\
 -2 & 4 & 1\\
 -3 & 5 & 3
\end{pmatrix}
\) 
の余因子行列\(\widetilde{A}\)を求めなさい.
% 7, 1, 2
% 3, 0, 1
% 2, 1, 0
\AMCOpen{lines=4, dots=false, framerule=0pt}{\wrongchoice[W]{誤}\scoring{0}\wrongchoice[P]{部}\scoring{1}\correctchoice[C]{正}\scoring{2}}
\end{question}
}
%%%%%%%%%%%%%%
\element{cof}{
\begin{question}{cof02}
行列
\(
A=\begin{pmatrix}
 1 & -1 & 2\\
 -1 & 3 & -5\\
 2 & 1 & -1
\end{pmatrix}
\) 
の余因子行列\(\widetilde{A}\)を求めなさい.
%   2,  1, -1
% -11, -5,  3
%  -7, -3,  2
\AMCOpen{lines=4, dots=false, framerule=0pt}{\wrongchoice[W]{誤}\scoring{0}\wrongchoice[P]{部}\scoring{1}\correctchoice[C]{正}\scoring{2}}
\end{question}
}
%%%%%%%%%%%%%%
\element{cof}{
\begin{question}{cof03}
行列
\(
A=\begin{pmatrix}
 1 & -2 & 1\\
 -6 & -1 & 1\\
 2 & -2 & 1
\end{pmatrix}
\) 
の余因子行列\(\widetilde{A}\)を求めなさい.
% 1, 0, -1
% 8, -1, -7
% 14, -2, -13
\AMCOpen{lines=4, dots=false, framerule=0pt}{\wrongchoice[W]{誤}\scoring{0}\wrongchoice[P]{部}\scoring{1}\correctchoice[C]{正}\scoring{2}}
\end{question}
}
%%%%%%%%%%%%%%
\element{cof}{
\begin{question}{cof04}
行列
\(
A=\begin{pmatrix}
 8 & -7 & 6\\
 -1 & 1 & -1\\
 -5 & 5 & -6
\end{pmatrix}
\) 
の余因子行列\(\widetilde{A}\)を求めなさい.
% -1, -12, 1
% -1, -18, 2
% 0, -5, 1
\AMCOpen{lines=4, dots=false, framerule=0pt}{\wrongchoice[W]{誤}\scoring{0}\wrongchoice[P]{部}\scoring{1}\correctchoice[C]{正}\scoring{2}}
\end{question}
}


\onecopy{2}{
%%% ヘッダー開始: %%%   
\noindent{\textbf{線形代数 演習11 \hfill 2019年7月4日}}

\vspace{3ex}

{\small
  \setlength{\parindent}{0pt}\hspace*{\fill}\AMCcodeGridInt{学生番号}{8}\hspace*{\fill}
\begin{minipage}[b]{6.5cm}
$\longleftarrow{}$\hspace{0ptplus1cm}
学生番号を左にマークし,下に氏名を記入してください.
%NU-AppsGのメールアドレス(brxxyyzzz)
%xxは名前から,yyは入学年度,zzzは学生番号下3桁です.@以降は必要ありません.

\vspace{3ex}

\hfill\namefield{
  \fbox{
    \begin{minipage}{.9\linewidth}
      氏名\vspace*{.5cm}%\dotfill\vspace*{.5cm}\dotfill
      \vspace*{1mm}
    \end{minipage}
  }
}
\hfill\vspace{5ex}
\end{minipage}
\hspace*{\fill}
}
% \multiSymbole{}の記号のある設問の正解は1個とは限りません。 0個の場合や複数の場合があります。

% 解答欄上部の「誤部正」は採点欄ですので,決して記入しないでください.
%解答欄に符号\((+, -)\)がある場合,正の数は\(+\),負の数は\(-\)を忘れずにマークすること.
%また,解答の数値が2桁の場合は,上が「十の位」,下が「一の位」です.

\vspace{1ex}
\hrulefill
\vspace{1ex}

%%% ヘッダー終了 %%%
%%%%%%%%%%%%%%%%%%%%%%%%%%%%%%%%%%%%%%%%%%%%%%%%%%%%%%%%%%%%%
\shufflegroup{cof}
\insertgroup[1]{cof}
} %end of \onecopy{}
\end{document}

